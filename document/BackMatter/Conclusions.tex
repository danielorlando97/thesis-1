\begin{conclusions}
      Se puede concluir que los resultados de la investigación realizada son favorables
      y que la implementación aportada es una buena solución para los objetivos planteados.
      Con la presente investigación se creó una herramienta con la que se puede describir
      detalladamente en lenguaje de alto nivel una amplia gama de espacios de búsqueda.
      Las descripciones que se pueden realizar con la explotación de la propuesta realizada
      son: 1) escalables y mantenibles, pues la sintaxis propuesta está basada en las
      clases de {\bf Python}, las descripciones resultantes cuentan con toda la flexibilidad y potencia
      del patrón orientado a objeto, 2) expresivas, pues se diseñó una sintaxis lo más
      similar posible a las definiciones de dominios matemáticos por lo que las descripciones resultantes suelen
      ser muy concretas y legibles, 3) independiente de los procesos y algoritmos de generación de
      muestra, se crearon las interfaces y los conectores necesarios para que los usuarios
      pudieran definir sus propias clases {\bf Sampler}.

      Mediante la experimentación se comprobaron las capacidades expresivas y generativas del 
      sistema propuesto. En los resultados de dichos 
      experimentos se evidenció que los procesos generativos del sistema son, en 
      general, eficientes en cuanto a tiempo, aunque nunca mejores que los generadores 
      imperativos que se pueden escribir con las herramientas del lenguaje. 
      Sobre dichos procesos generativos se demostró además su efectividad al generar muestras pertenecientes a los 
      espacios descritos. Por otro lado, las capacidades expresivas del sistema que se demostraron en dichos 
      experimentos son impresionantes. A lo largo del documento se mostraron decenas de ejemplos 
      que evidenciaron la expresividad de 
      la propuesta y la amplia gama de casos de usos en las que esta puede ser útil.

      De forma general se presentó una potencial solución a las limitaciones expresivas de {\bf AutoGOAL}.
      El sistema es mucho más interesante para desarrollos puntuales de
      soluciones de problemas de búsqueda que para un sistema del {\bf AutoML}, donde el tiempo de evaluación
      de la función objetivo es mucho más grande que el tiempo de generación de muestras.

      El resultado final es un {\bf DSL} muy expresivo y coherente en la generación de muestra. Dicho {\bf DSL} es
      capaz de representar expresiones que hasta el momento prácticamente no tenían representación simple
      ni declarativa. Entre la lista de relaciones descriptibles que hacen que el {\bf DSL} propuesto es un
      salto evolutivo en la descripción de espacio de búsqueda con respecto al estado del arte se
      encuentran las dependencias contextuales, las dependencias condicionales y los tensores de
      dimensiones dinámicas.

      El uso de la herramienta propuesta puede suponer un salto importante en la
      legibilidad del código. Además de aumentar la productividad a la hora de resolver problemas de
      búsqueda pues una mejor descripción del espacio de búsqueda se traduce en una mejor comprensión
      de la estructura del mismo. Dicha comprensión no solo es importante para el desarrollador en
      cuestión a la hora de diseñar la soluciones o pensar las heurísticas, sino que además facilita
      el análisis del código en cuestión por personas ajenas al desarrollo.
\end{conclusions}
