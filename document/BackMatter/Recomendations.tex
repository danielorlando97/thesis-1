\begin{recomendations}
      La presente investigación se realizó dentro de un marco temporal limitado. 
      Por tanto, al final de la misma no se pudieron resolver todos los problemas 
      que fueron apareciendo durante el desarrollo. Muchos de estos problemas son 
      funcionalidades o refactorizaciones que pudieran mejorar el sistema propuesto, 
      ya sea en eficiencia o en expresividad. Todas aquellas ideas que surgieron durante 
      la presente investigación y no se contó con el tiempo suficiente para llevarlas a 
      cabo, se enumeran a continuación como recomendaciones para trabajos futuros: 

    \begin{enumerate}
        \item   Persistencia del aprendizaje resultante del proceso de “prueba y error” para los casos
              simples. Como la propuesta de solución separa en clases distintas cada uno de los casos
              específicos, entonces se puede describir un nuevo “método mágico” que represente el proceso
              posterior a la detección de errores. Donde en los casos de los tipos simples como enteros o
              categorías se modifique el dominio inicial, pues en estos casos el dominio no es dinámico pues
              no existe dependencias contextuales. Y, por otro lado, las clases más complejas como listas,
              clases o espacios sensibles al contexto implementen la identidad en dicho método. Se
              recomienda además almacenar el dominio inicial previendo los casos en que las restricciones
              se adicionen de forma dinámica entre generación y generación, en dicho casos puede ocurrir
              una mutación del espacio en alguna generación.
        \item  Desarrollo de una colección de funciones de caja negra. Aprovechando la sintaxis de
              caja negra se puede desarrollar varias colecciones de funciones básicas, aportando a la biblioteca
              mayor expresividad y potencialmente eficiencia. Entre las principales ideas se proponen las
              siguientes colecciones: 1) definiciones matemáticas (primo, factorial, raíz cuadrada, .....),
              2) operaciones textuales (sufijo, prefijo, .....) y 3) operaciones con listas o tensores (máximo,
              mínimo, media, ....).
        \item  Investigación para describir el proceso inverso de una determinada función de caja negra,
              para aportar información al proceso de inferencia de dominios en los casos en que dicha función
              depende de un elemento cualquiera del espacio subyacente. La biblioteca con el dominio de
              transformaciones lineales resuelve las descripciones inapropiadas que se escriben mediante
              operaciones definidas en la gramática de forma tal que la variable principal de la función de
              restricciones no aparezca despejada. En el caso de las funciones de "caja negra" cuando existe
              dependencia de esta hacia la variable principal simplemente se ignora dicha restricción en
              fase de inferencia, pero muchas funciones que se describirían como funciones de caja negra
              cuentan con una inversa conocida. Véase por ejemplo la función raíz cuadrada, si la función de
              restricciones destaca que la raíz cuadrada de {\bf x} es menor que 10 eso implica necesariamente
              que {\bf x} es menor que 100. Resolver este problema unido a la colección de funciones propuestas
              anteriormente puede representar una gran optimización para el DSL tanto en expresividad como en
              generación.

              \begin{listing}[!ht]
                  \begin{minted}{Python}
      import math
      class Sqrt(FunctionalConstraint):
            def __call__(self, x: int):
                  return math.sqrt(x)

            def __inverse__(self, x: int):
                  return x * x

      space = Domain[int](min = 0) | (lambda x: Sqrt(x) < 10)
      print(space.limits) # (0, 100)
                  \end{minted}
                  \caption{Ejemplo de recomendación para disminuir los casos de restricciones de ``caja negra'' sin participación en el procesos de reducción de dominios}
              \end{listing}

        \item   Restricciones que afecten a más de un espacio. El modelo seleccionado para la representación
              de las listas y los tensores es la de una clase superior que referencia a cada uno de los espacios
              individuales. La herramienta cuenta con los mecanismos necesarios para trasmitir las restricciones
              individuales a cada uno de los espacios internos, pero no cuenta con la infraestructura necesaria
              para describir dependencias entre listas o tuplas. Ósea, sea A un tensor que simula una lista de
              tuplas de tamaño dos, no existe ningún mecanismo para describir que el índice 0 de dicha lista es
              igual a la tupla (1, 2) por ejemplo. Esto se puede lograr definiendo un nuevo dominio ordinal que
              referencie a los dominios internos y pueda distribuir las restricciones. Para detectar dichos casos
              se debe aplicar una refactorización al recorrido implementado para detectar las operaciones de
              indexación.
      \begin{listing}[!ht]
            \begin{minted}{Python}
            Domain[int][10] | (lambda x: x[0:2] = (1, 2))
            Domain[int][10][2] | (lambda x, i: x[i] = (i, i+1))
            \end{minted}
            \label{nada}
            \caption{Ejemplo de recomendación aplicar restricciones que afecten a más de un espacio a la vez}
      \end{listing}
      \newpage
      \item La clase {\bf LinearTransformedDomain}, implementada como parte de la jerarquía de clases que describen 
      el álgebra de dominios definida, es una de las últimas ideas incorporadas a la propuesta de solución. Debido a 
      dicha incorporación tardía, dicha funcionalidad al final de la presente investigación aún contaba con algunas
      limitaciones. 
      
      \begin{listing}[!ht]
            \begin{minted}{Python}
            Domain[int][10] | (lambda x: (x % 10 > 0, x % 10 < 5))
            Domain[int][10] | (lambda x: (x % 10 % 10 < 5))
            \end{minted}
            \caption{Ejemplo de recomendación aplicar restricciones que afecten a más de un espacio a la vez}
            \label{rec:mod}
      \end{listing}

      La propuesta de solución no cuenta con las herramientas necesarias para detectar que en el primer ejemplo de 
      \ref{rec:mod} en ambas restricciones se referencia a la misma transformación. Al final de la presente 
      investigación, los dos ejemplos presentados en \ref{rec:mod} son análogos en cuanto a la cantidad y forma de las 
      transformaciones. Para detectar dicha similitud y diferencias es necesario que al momento de solicitar una 
      transformación se suministre una noción de la composición del fragmento de AST que está provocando dicha 
      transformación. Además, los dominios de transformaciones lineales deben conocer cuál es la estructura que 
      provoco su creación. Finalmente, se necesita tener conciencia sobre el concepto de conmutabilidad a la 
      hora de comparar dichas estructuras.



    \end{enumerate}
\end{recomendations}
