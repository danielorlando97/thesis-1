\begin{resumen}
  {\bf AutoGOAL} es una de las bibliotecas del estado del arte que analiza la
  problemática del {\it AutoML}. Esta se distingue del resto por ser transversal
  a la naturaleza del problema, característica que es consecuencia directa
  de su elección y diseño de la estrategia de búsqueda,
  {\it Evolución Gramatical Probabilística}
  ({\it Probabilistic Grammatical Evolution}, {\bf PGE}).

    {\bf PGE} realiza el proceso de búsqueda y optimización a partir de una
  gramática previamente definida. Por tanto, mientras mejor sea la
  capacidad de la biblioteca para describir espacios de búsqueda y generar
  sus respectivas gramáticas, más amplio será el conjunto de los problemas
  que esta podrá intentar resolver. Las herramientas existentes previas al
  desarrollo de la presente tesis se limitaban a la definición e inferencia
  de gramáticas libres del contexto

  En este documento se presenta una nueva herramienta, que al integrarse con
  \newline{\bf AutoGOAL} podría ampliar el poder descriptivo de este hasta la
  generación de gramáticas sensibles al contextos sin dependencias circulares.
  Dicha nueva biblioteca define un {\bf DSL}
  ({\it Lenguaje de Dominio Específico}, {\it Domain Specific Language}) capaz de
  describir, con una filosofía {\it Bottom-Up}, la composición de los
  espacios de búsqueda y las relaciones y restricciones entre sus
  componentes internos, para posteriormente generar muestras apoyándose en
  dichas descripciones

\end{resumen}

\begin{abstract}

  {\bf AutoGOAL} is one of the state-of-the-art libraries analyzing the {\bf AutoML} issue.
  It is distinguished from the rest by being transversal to the nature of the problem, a
  characteristic that is a direct consequence of its choice and design of the search strategy,
  {\bf Probabilistic Grammatical Evolution} ({PGE}).

    {\bf PGE} performs the search and optimization process from a previously defined grammar.
  Therefore, the better the library's ability to describe search spaces and generate their
  respective grammars, the broader the set of problems it can attempt to solve. Existing tools
  prior to the development of this thesis were limited to the definition and inference of
  context-free grammars.

  This paper presents a new tool, which when integrated with {\bf AutoGOAL} could extend the
  descriptive power of {\bf AutoGOAL} to the generation of context-sensitive grammars without
  circular dependencies. This new library defines a {\bf DSL} ({\bf Domain Specific Language})
  capable of describing, with a {\bf Bottom-Up} philosophy, the composition of the search spaces
  and the relationships and constraints of the search spaces. and the relationships and constraints
  between its internal components, in order to subsequently internal components, and then generate
  samples based on these descriptions. these descriptions

\end{abstract}