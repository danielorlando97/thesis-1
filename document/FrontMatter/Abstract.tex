\begin{resumen}
  {\bf AutoGOAL} es una de las bibliotecas del estado del arte que analiza la
  problemática del {\it AutoML}. Esta se distingue de las alternativas por ser transversal
  a la naturaleza del problema, característica que es consecuencia directa
  de su elección y diseño de la estrategia de búsqueda,
  {\it Evolución Gramatical Probabilística}
  ({\it Probabilistic Grammatical Evolution}, {\bf PGE}).

    {\bf PGE} realiza el proceso de búsqueda y optimización a partir de una
  gramática previamente definida. Por tanto, mientras mejor sea la
  capacidad de la biblioteca para describir espacios de búsqueda y generar
  sus respectivas gramáticas, más amplio será el conjunto de los problemas
  que esta podrá intentar resolver. Las herramientas existentes previas al
  desarrollo de la presente tesis se limitaban a la definición e inferencia
  de gramáticas libres del contexto.

  En este documento se presenta una nueva herramienta, que al integrarse con
  \newline{\bf AutoGOAL} podría ampliar el poder descriptivo de este hasta la
  generación de gramáticas sensibles al contexto sin dependencias circulares.
  Dicha nueva biblioteca define un {\bf DSL}
  ({\it Lenguaje de Dominio Específico}, {\it Domain Specific Language}) capaz de
  describir, con una filosofía {\it Bottom-Up}, la composición de los
  espacios de búsqueda y las relaciones y restricciones entre sus
  componentes internos, para posteriormente generar muestras apoyándose en
  dichas descripciones.

  La investigación que se describe en el documento finaliza sometiendo al 
  sistema propuesto a una serie de experimentos que intentan evaluar las 
  capacidades expresivas y generativas del mismo. En los resultados de dichos 
  experimentos se evidencia que los procesos generativos del sistema son, en 
  general, eficientes en cuanto a tiempo, aunque nunca mejores que un generador 
  imperativo escrito con las herramientas del lenguaje. Sobre dichos procesos generativos se 
  demuestra además su efectividad al generar muestras pertenecientes a los 
  espacios descritos. 

  Por otro lado, a lo largo del documento y como parte de los 
  experimentos se muestran decenas de ejemplos que evidencian la expresividad de 
  la propuesta y la amplia gama de casos de usos en las que esta puede ser útil. 
  Concluyendo que la misma es un salto evolutivo en cuanto a la expresividad de 
  las descripciones de los espacios de búsqueda, llegando a ser despreciable las 
  diferencias en cuanto a tiempo de generación que pueda tener con herramientas menos expresivas.

\end{resumen}

\begin{abstract}

  {\bf AutoGOAL} is one of the state-of-the-art libraries analyzing the {\bf AutoML} issue.
  It is distinguished from the rest by being transversal to the nature of the problem, a
  characteristic that is a direct consequence of its choice and design of the search strategy,
  {\bf Probabilistic Grammatical Evolution} ({PGE}).

    {\bf PGE} performs the search and optimization process from a previously defined grammar.
  Therefore, the better the library's ability to describe search spaces and generate their
  respective grammars, the broader the set of problems it can attempt to solve. Existing tools
  prior to the development of this thesis were limited to the definition and inference of
  context-free grammars.

  This paper presents a new tool, which when integrated with {\bf AutoGOAL} could extend the
  descriptive power of {\bf AutoGOAL} to the generation of context-sensitive grammars without
  circular dependencies. This new library defines a {\bf DSL} ({\bf Domain Specific Language})
  capable of describing, with a {\bf Bottom-Up} philosophy, the composition of the search spaces
  and the relationships and constraints of the search spaces. and the relationships and constraints
  between its internal components, in order to subsequently internal components, and then generate
  samples based on these descriptions. these descriptions.

  The research described in the paper concludes by subjecting the proposed system to a series of 
  experiments that attempt to evaluate the expressive and generative capabilities of the system. 
  The results of these experiments show that the generative processes of the system are, 
  in general, efficient in terms of time, although never better than an imperative generator 
  written with the language tools. The effectiveness of these processes in generating samples 
  belonging to the described spaces is also demonstrated. 
  
  On the other hand, the expressive capabilities of the system demonstrated in these experiments 
  are impressive. Throughout the document and as part of the experiments, dozens of examples are 
  shown that demonstrate the expressiveness of the proposal and the wide range of use cases in 
  which it can be useful. Concluding that it is an evolutionary leap in terms of the expressiveness 
  of the descriptions of the search spaces, becoming negligible the differences in terms of 
  generation time that this may have with the imperative generators. 

\end{abstract}