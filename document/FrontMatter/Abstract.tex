\begin{resumen}
    {\bf AutoGOAL} es una de las bibliotecas del estado del arte que analiza la
    problemática del {\it AutoML}. Esta se distingue del resto por ser transversal
    a la naturaleza del problema, característica que es consecuencia directa
    de su elección y diseño de la estrategia de búsqueda,
    {\it Evolución Gramatical Probabilística}
    ({\it Probabilistic Grammatical Evolution}, {\bf PGE}).

        {\bf PGE} realiza el proceso de búsqueda y optimización a partir de una
    gramática previamente definida. Por tanto, mientras mejor sea la
    capacidad de la biblioteca para describir espacios de búsqueda y generar
    sus respectivas gramáticas, más amplio será el conjunto de los problemas
    que esta podrá intentar resolver. Las herramientas existentes previas al
    desarrollo de la presente tesis se limitaban a la definición e inferencia
    de gramáticas libres del contexto

    En este documento se presenta una nueva herramienta, que al integrarse con
        {\bf AutoGOAL} podría ampliar el poder descriptivo de este hasta la
    generación de gramáticas sensibles al contextos sin dependencias circulares.
    Dicha nueva biblioteca define un {\bf DSL}
    ({\it Lenguaje de Dominio Específico}, {\it Domain Specific Language}) capaz de
    describir, con una filosofía {\it Bottom-Up}, la composición de los
    espacios de búsqueda y las relaciones y restricciones entre sus
    componentes internos, para posteriormente generar muestras apoyándose en
    dichas descripciones

\end{resumen}

\begin{abstract}
    Resumen en inglés
\end{abstract}