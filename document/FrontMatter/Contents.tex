\tableofcontents
% \listoffigures
% \listoftables
% \listofalgorithms

\chapter*{Ejemplos de Código}\label{chapter:examples}

\begin{listing}[!ht]
    \begin{minted}{Python}
class Line:
    def __init__(
        self,
        m: int = Domain[int](min=50, max=100) | (
            lambda x: x != 65),
        n: float = Domain[float]() | (
            lambda x: x < 50)
    ) -> None:
        self.m, self.n = m, n
    \end{minted}
    \caption{Espacio de las rectas}
    \label{lst:rectas}
\end{listing}


\begin{listing}[!ht]
    \begin{minted}{Python}
class CenterPoint:
    X_Domain = Domain[int](min=0, max=100)

    def __init__(
        self,
        x: int = X_Domain,
        y: float = Domain[float] | (
            lambda y, x=X_Domain: (x - 10 < y, y < x + 10)
        ),
    ) -> None:
        self.x, self.y = x, y
    \end{minted}
    \caption{Espacio de los puntos acotados por las rectas y = x - 10 y y = x + 10}
    \label{lst:points}
\end{listing}

\begin{listing}[!ht]
    \begin{minted}{Python}
class LogisticRegression:
    Solver_Domain = Domain[str](
        options=['newton-cg', 'lbfgs', 'liblinear', 'sag', 'saga'])

    def __init__(
        self,
        solver: str = Solver_Domain,
        intercept_scaling: float = Domain[float] | (
            lambda x, s=Solver_Domain: 
                (s != 'liblinear') & (x == 1)
        ),
        random_state: int = Domain[Optional[int]] | (
            lambda x, s=Solver_Domain: (s != [
                'liblinear', 'sag', 'saga']) & (x == None)
        )
    ) -> None:
        self.s, self.i = solver, intercept_scaling
        self.r = random_state
    \end{minted}
    \caption{Sklearn LogisticRegression}
    \label{lst:sklearn}
\end{listing}

\begin{listing}[!ht]
    \begin{minted}{Python}
@FunctionalConstraint
def list_div(x: int):
    result = [i for i in range(2, x)]
    for i in range(2, x):
        if x % i == 0:
            result = [j for j in result 
                        if j % i != 0]
    return result
class WorkPlanning:
    Review_Work_Domain = Domain[int](min=2, max=10000)
    def __init__(self,
        init_time: int = Domain[int] | (
            lambda x, y=Review_Work_Domain: (
                x == DivList(y), x > 0
            )
        ),
        review_time: int = Review_Work_Domain,
    ) -> None:
        self.init, self.review = init_time, review_time
    \end{minted}
    \caption{Espacio de parejas de enteros primos relativos}
    \label{lst:primos}
\end{listing}

\newpage
\begin{listing}[!ht]
    \begin{minted}{Python}
class GraphByAdjMatrix:
NDomain = Domain[int](min=0, max=1000)
AdjMatrixDomain = Domain[bool][NDomain][NDomain] | (
    lambda x, i, j: x[i][j] == x[j][i]
)
def __init__(self, n: int = NDomain,
    adj_matrix: List[int] = AdjMatrixDomain
) -> None:
    self.n, self.matrix = n, adj_matrix
    \end{minted}
    \caption{Espacio de grafos, modelados por matriz de adyacencia}
    \label{lst:adj}
\end{listing}


\begin{listing}[!ht]
    \begin{minted}{Python}
class Node:
RightDomain = Domain[Optional[Self]]()
LeftDomain = Domain[Optional[Self]]() | (
    lambda x: x.LeftDomain == None
)
def __init__(self, right: Self = RightDomain,
    left: Self = LeftDomain
) -> None:
    self.right, self.left = right, left
    \end{minted}
    \caption{Espacio de grafos, modelados orientados a objeto}
    \label{lst:node}
\end{listing}


\begin{listing}[!ht]
    \begin{minted}{Python}
class BagItem:
WeightDomain = Domain[float]()
PriceDomain = Domain[float](max=50)
def __init__(self, w: float = WeightDomain, 
    p: float = PriceDomain
) -> None:
    self.w, self.p = w, p

class BagProblem:
    WeightDomain = Domain[float](max=100)
    ItemsLenDomain = Domain[int](min=5, max=20)
    ItemsDomain = Domain[BagItem][ItemsLenDomain] | (
        lambda x, i, w=WeightDomain: x[i].WeightDomain < w
    )
    def __init__(self, w: float = WeightDomain, 
        items: List[BagItem] = ItemsDomain
    ) -> None:
        self.w, self.items = w, items
    
@FunctionalConstraint
def ComputingCurrentCapacity(w: float, items: List[BagItem], 
    current_counts: List[float]
):
    current_w = sum([count * item.w for count,
                    item in zip(current_counts, items)])
    return (w - current_w) / items[len(current_counts)].w

class BagSolution(BagProblem):
    SolutionDomain = Domain[int][BagProblem.ItemsLenDomain] | (
        lambda x, i, w=BagProblem.WeightDomain, items=BagProblem.ItemsDomain: (
            x[i] < ComputingCurrentCapacity(w, items, x[:i - 1])
        ))
    def validate_solution(self, solution: List[int] = SolutionDomain):
        price = sum([item.p * count for item,
                     count in zip(self.items, solution)])
        weight = sum([item.w * count for item,
                     count in zip(self.items, solution)])
        assert self.w >= weight
        return price

    \end{minted}
    \caption{La mochila}
    \label{lst:bag}
\end{listing}



\begin{listing}[!ht]
    \begin{minted}{Python}
@FunctionalConstraint
def AdjColors(_map: List[List[bool]], 
    country_colored: List[int], 
    color_len
):
    index = len(country_colored)
    result = set()
    for i, is_adj_country in enumerate(_map[index]):
        if i < index and is_adj_country:
            result.add(country_colored[i])
    if len(result) == color_len:
        return []
    return list(result)

class ColorMapProblem:
    CountryLenDomain = Domain[int](min=5, max=20)
    ColorLenDomain = Domain[int](min=1) | (
        lambda x, cl=CountryLenDomain: x < cl)

    MapDomain = Domain[bool][CountryLenDomain][CountryLenDomain] | (
        lambda x, i, j: x[i][j] == x[j][i]
    )
    def __init__(self, color_len: int = ColorLenDomain, 
        adj_map: List[List[bool]] = MapDomain
    ) -> None:
        self.cl, self.map = color_len, adj_map

    SolutionDomain = Domain[int][CountryLenDomain] | (
        lambda x, i, color_len=ColorLenDomain, _map=MapDomain: (
            x[i] < color_len,
            x[i] != AdjColors(_map, x[0:i-1], color_len)
        )
    )
    def validate_solution(self, solution: List[int] = SolutionDomain):
        assert len(solution) == len(self.map)
        for color in solution:
            assert color < self.cl

        for i, row in enumerate(self.map):
            for j, are_adj in enumerate(row):
                if are_adj and solution[i] == solution[j]:
                    return False
        return True
    \end{minted}
    \caption{N Colores}
    \label{lst:colors}
\end{listing}


\begin{listing}[!ht]
    \begin{minted}{Python}
class AbsCluster:
    NClusterDomain = Domain[int]()
    def __init__(self, n_clusters: int = NClusterDomain) -> None:
        self.n = n_clusters

class KMeans(AbsCluster):
    pass

class AgglomerativeClustering(AbsCluster):
    def __init__(
        self,
        n_clusters: int = AbsCluster.NClusterDomain,
        affinity: str = Domain[str](
            options=['euclidean', 'manhattan', "l1", "l2"]
        )
    ) -> None:
        super().__init__(n_clusters)
        self.affinity = affinity

class AbsClassifier:
    NCategoryDomain = Domain[int]()
    def __init__(self, n_categories: int = NCategoryDomain) -> None:
        self.n = n_categories

class KNN(AbsClassifier):
    def __init__(self, n_categories: int = AbsClassifier.NCategoryDomain) -> None:
        super().__init__(n_categories)
        k = self.n * 2 - 1

class GaussianNB(AbsClassifier):
    def __init__(self, n_categories: int = AbsClassifier.NCategoryDomain) -> None:
        super().__init__(n_categories)
        p = [1/n_categories for _ in range(n_categories)]

class SemiSupervisedClassifier:
    ClusterDomain = Domain[Union[KMeans, AgglomerativeClustering]]()
    def __init__(self, cluster: AbsCluster = ClusterDomain,
        classifier: AbsClassifier = Domain[Union[KNN, GaussianNB]] | (
            lambda x, c=ClusterDomain: x.NCategoryDomain == c.NClusterDomain)
    ) -> None:
        self.cluster, self.classifier = cluster, classifier
    \end{minted}
    \caption{AutoML,Modelo Semi Supervizado}
    \label{lst:automl}
\end{listing}