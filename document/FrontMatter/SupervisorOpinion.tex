\begin{opinion}
    En la actualidad el Aprendizaje Automático ha llegado a todas las ramas de la industria, 
    ayudando a resolver un gran número de problemas pero creando la necesidad de un enorme 
    número de expertos para poder utilizar las herramientas adecuadas en cada caso.
    En este escenario el AutoML propone una solución ayudando con la selección de forma 
    automática de las mejores soluciones con el problema añadido de que incrementa
    el costo computacional ya que tiene que evaluar muchas soluciones para resolver cada 
    problema. El área de investigación en que incursiona el estudiante se relaciona con 
    la descripción de los espacios de búsqueda, y propone un enfoque para permitir a los 
    usuarios de sistemas AutoML definir un espacio de forma mucho más expresiva y concisa.
    
    El estudiante Daniel Orlando Ortiz Pacheco en esta investigación se adentró en un tema 
    del estado del arte de gran actualidad y para eso tuvo que utilizar conocimientos de 
    varias asignaturas de la carrera y otros que no son parte del currículum estándar. 
    Su propuesta implicó la definición de un DSL embebido en Python que permite expresar 
    espacios de variables aleatorias de estructura arbitraria, con parámetros condicionales, 
    dependencias contextuales entre los parámetros, todo esto garantizando la máxima eficiencia 
    posible durante la generación a partir de un creativo mecanismo para inferir y explotar 
    la estructura sintáctica de las relaciones entre los parámetros.
    
    Sus resultados resultan muy prometedores, pues permiten definir de forma declarativa espacios 
    de búsqueda con estructuras muy complejas que hasta el momento solo es posible en la literatura 
    a través de definiciones imperativas. Aunque sus resultados llegan hasta la definición e 
    implementación del lenguaje y el mecanismo de muestreo, su inclusión en un sistema de AutoML 
    es directa. Además, el DSL puede ser aplicado a problemas de búsqueda muy variados, más 
    allá del campo del AutoML.
    
    Para poder afrontar el trabajo, el estudiante tuvo que revisar literatura científica 
    relacionada con la temática así como soluciones existentes y bibliotecas de software 
    que pueden ser apropiadas para su utilización. Todo ello con sentido crítico, determinando 
    las mejores aproximaciones y también las dificultades que presentan.
    
    Todo el trabajo fue realizado por el estudiante con una elevada constancia, capacidad de 
    trabajo y habilidades, tanto de gestión, como de desarrollo y de investigación.
    Por estas razones pedimos que le sea otorgada al estudiante Daniel Orlando Ortiz Pacheco 
    la máxima calificación y, de esta manera, pueda obtener el título de Licenciado 
    en Ciencia de la Computación
    
    
    

\begingroup
  \centering
  \wildcard{Lic. Frank Sadan Naranjo}
  \hspace{1cm}
  \wildcard{Dr.C. Alejandro Piad Morffis}
  \par
\endgroup

\end{opinion}
