\begin{acknowledgements}
    Ahora que está por finalizar mi experiencia en la facultad de Matemática y Computación (Matcom) 
    de la Universidad de la Habana, no puedo dejar de pensar que cuando decidí dejar la facultad de 
    Física Nuclear para embarcarme en esta aventura tomen la mejor decisión de mi vida. Hoy puedo 
    confirmar que en aquellas primeras clases de C++ de la facultad de Física encontré mi verdadera 
    vocación. En estas líneas quiero agradecer a todas aquellas personas que de alguna manera u otra 
    hayan contribuido a mi formación, a mi crecimiento y a mi felicidad.
    
    A mi familia que siempre me lo ha dado todo sin pedir nada a cambio, que siempre han estado ahí
    para mí cuando los he necesitado. Sin ellos hubiese sido muy difícil este camino de dedicación y 
    esfuerzo.
    
    A aquellos amigos que hablan de mí como "el que estudia la locura esa", mi válvula de escape, 
    los que me recuerdan día a día que la amistad no va de intereses en común, sino de momentos, de 
    historias, de risa y lágrimas.

    A todos mis compañeros de Matcom, un colectivo que brilla mucho más por su calidad humana que por 
    si infinita inteligencia, creatividad y perseverancia. Con los que he compartido estos 5 años 
    maravilloso y a los que les deseo de corazón los más grandes éxitos que puedan existir en 
    la galaxia.

    A mis tutores, que han tenido confianza en mí desde el primer momento. Que han tenido la paciencia
    para leerse todo lo que se me ocurrían escribir. A los que siempre les enseñe mi trabajo con 
    humildad pensando "para todo lo se saben, esto que les estoy contando les debe dar básicamente 
    igual" y pese a su amplia experiencia siempre respondieron con entusiasmo ante cada resultado. 
    Que siempre ha estado disponibles y dispuestos a ayudar. A ambos muchas gracias.

    A mis compañeros de la Facultad de Física Nuclear, los que me enseñaron a estudiar, a los que les
    pertenece más de la mitad de mis notas en esa carrera. Aunque nuestros caminos no se crucen con 
    frecuencia, siempre recordaré como me acogieron cuando la probabilidad de que fuéramos incompatibles 
    era máxima.

    A todos los compañeros que he tenido en mis experiencias laborales, que muchas veces han tenido que 
    responder por mí cuando yo he tenido quehaceres de la escuela. Ustedes han sido sin duda una escuela 
    más.

    A todos muchas gracias, no creo tener la oportunidad de devolverles todo lo que me han dado, pero 
    siempre que lo necesiten estaré ahí.
    
    Mil gracias, los quiero mucho .....
\end{acknowledgements}