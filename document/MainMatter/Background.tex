\chapter{Estado del Arte}\label{chapter:state-of-the-art}


El espacio de búsqueda es una de las componentes principales de los sistemas
de {\it AutoML}, independientemente de la naturaleza de los mismo. La gran mayoría
de las herramientas de sector cuentan con una espacio de búsqueda muy específico,
normalmente determinado por el tipo de problema al que dan solución o a las
herramientas de {\it ML} subyacentes. En dichos casos, se llama "{\it descripción
del espacio de búsqueda}" a la delimitación de los hyperparámetros de los
distintos modelos que se incluyen en dicha definición previa

Como se muestra en las investigaciones previas realizadas por [5] casi ninguna
de las herramientas del sector {\it AutoML} cuenta con la capacidad real de describir,
en su totalidad y de forma detallada, su espacio de búsqueda. Independientemente
de esto, todo mecanismo y sintaxis como el objetivo de describir estructuralmente
ciertos dominios, para si posterior explotación, pueden resultar de interés para
el procesos investigativo de la presente tesis.

\section{Marco Teórico}

\subsection{Espacio de Búsqueda}

Subconjunto del universo tal que cada uno de sus elementos son soluciones factibles
para una problema dado.

\subsection{Descripción del Espacio de Búsqueda}


Proceso mediante el cual, según las características del medio, se define una
jerarquía de estructuras y datos de forma tal que el receptor de dicha descripción
pueda entender la estructura, dimension y características del espacio y sea capaz
de generar muestras del mismo.

En el caso específico del sector computacional el medio no suele ser un lenguaje
de programación de uso general, sino que las distintas herramientas de optimización,
{\it ML} o {\it AutoML}, suelen implementar un {\bf DSL's} o {\it frameworks} en los
que las descripciones se pueda acercar un poco más al lenguaje natural.

\subsection{DSL (Lenguaje de Dominio Específico, Domain Specific Language)}

Según [6] un lenguaje de dominio específico es un lenguaje especializado que sirve
para elevar el nivel de abstracción del software y facilitar el desarrollo del mismo.
Los {\bf DSL} cuentan con múltiples formas de representación e implementación, desde
micro-modificaciones realizadas a lenguajes subyacentes, hasta proyectos a gran escala.

Para la presente investigación solo son de interés aquellos {\bf DSLs} que se construyen e
incorporan a un lenguaje subyacente, elevando el nivel de abstracción y la expresividad
del código con respecto a un domino específico.

\subsection{DSLs orientados a descripciones}

Los {\bf DSLs} pertenecientes al domino de la presente investigación tiene como objetivo
la descripción de una serie de conceptos para su posterior explotación. Esto supone que
los mismos cuenten con una ejecución separada en dos fases, que puede recordar a las
tradicionales etapas de los programas ({\it compilación} y {\it ejecución}), pues en un
primer momento los desarrolladores escriben toda una serie de reglas y descripciones
que posteriormente serán utilizadas para la ejecución o ejecuciones en búsqueda de un
propósito final.

Debido al hecho de que estas herramientas, como se señaló anteriormente, representa un
pequeño engranaje en sistemas mucho más grandes, entonces generalmente las dos fases
antes descritas suelen tener lugar en algún momento de la ejecución del sistema como un
todo. Por tanto depende en gran medida de las características del sistema general, el
que se pueda pensar en estos '{\it componentes descriptivos}' como una herramienta de dos
fases o en un único componente

Precisamente los sistemas que son objeto de estudio de esta tesis presentan las
características necesarias para pensar en sus {\bf DSLs} como mecanismos de dos fases.
Esos sistemas de búsquedas, {\it ML} y {\it AutoML} suelen tener un alto costo temporal de
explotación e investigación del espacio. Por dicha razón podemos considerar todas las
declaraciones e instancias iniciales necesarias para orquestar la búsqueda como un
costo fijo, y luego tener un segundo análisis temporal sobre la dilatación de dicho
sondeo del elemento optimo.

En estos casos específicos el autor considera razonable hablar de las dos fases de
estos {\bf DSLs} como {\it tiempo de compilación del DSL} y
    {\it tiempo de ejecución del DSL}

\subsection{Tiempo de compilación del DSL}


Definimos el {\it tiempo de compilación de un DSL} como todas las operaciones puntuales que se
realizan para orquestar la infraestructura que dará soporte a la ejecución del objetivo
final del mismo. Dicha infraestructura debería permanecer inmutable en su mayoría
durante todos los procesos posteriores a la "{\it compilación}" del {\bf DSL}.

\subsection{Tiempo de ejecución del DSL}


Definimos el {\it tiempo de ejecución del DSL} como todas las operaciones que realiza
la herramienta, posteriores a la "compilación" del mismo, para lograr su objetivo básico y
principal.

Véase por ejemplo el proceso de ejecución del módulo {\it Grammar} de {\bf AutoGOAL}, el cual es
la componente descriptiva del sistema. En un primer momento se analizan todas las descripciones
aportadas para inferir una gramática libre del contexto que describa el espacio de todos los
programas factibles, proceso que se pudiera interpretar como la "{\it compilación del sistema}". Y
posteriormente, de forma iterativa, se generan nuevas instancias de dicha gramática para ser
evaluadas y realizar otras operaciones ajenas al componente descriptivo. Si se interpreta que
el objetivo final del módulo {\it Gramar} es la generación de soluciones factibles, entonces se
podrían decir que una vez que inicia una ejecución de {\bf AutoGOAL}, luego de la "{\it compilación}"
de sus descripciones, su {\bf DSL} se ejecuta múltiples veces hasta que el sistema encuentra una
respuesta al problema planteado.

\subsection{Python para DSL}

Como el resultado de la presente tesis es un {\bf DSL} atado a un lenguaje de propósito general
subyacente. Dicho lenguaje debe contar con una serie de características especiales,
que permitan al autor modificar la semántica de su sintaxis original. Y aunque la gran mayoría
de los sistemas de la actualidad que podrían estar interesados en la explotación de la solución
propuesta están escritos en {\it Python}. La elección de dicho lenguaje como lenguaje subyacente
para la implementación de la propuesta planteada no se encuentra influenciada solo por dicha
situación, sino que además {\it Python} cuenta con potentes y cómoda herramientas para modificar
la semántica de su sintaxis y desarrollar la metaprogramación.

Según [14], {\it Python} es un lenguaje de programación de alto nivel y de propósito general.
    {\it Python} es de tipado dinámico y fuerte, pero existen bibliotecas, como {\bf typing}, que haciendo
uso de la metaprogramación y otras técnicas, son capaces de modificar la semántica del
lenguaje para ofrecer una experiencia de usuario similar a la de los lenguajes estéticamente
tipados.

Una de las características que hacen a este lenguaje tan flexible e ideal para la metaprogramción,
es la filosofía bajo la que describe todos los procesos y transiciones de sus objetos. El
interprete del lenguaje define una lista de "{\it métodos mágicos}", uno por cada operación básica
de software, los cuales cuanta con sus propias implementaciones básicas, pero que pueden ser
redefinidos en todo momento. Como señala [14], los "{\it métodos mágicos}" son los que comienzan y
terminan con el doble guión bajo, entre los que se pueden citar por ejemplo;
{\it \_\_and\_\_}, {\it \_\_sub\_\_}, {\it \_\_div\_\_}, {\it \_\_mul\_\_} y otra larga lista de funciones que hacen referencia
a las operaciones aritméticas y de comparación, {\it\_\_call\_\_} método que describe el comportamiento
de un objeto cuando se intenta usar como función, {\it \_\_getattribute\_\_} o {\it \_\_getitem\_\_}
que describen los comportamientos cuando se intenta acceder a los miembros de una clase o a
un indice determinado respectivamente. La lista es extremadamente larga y crece a medida
que aparecen nuevos bibliotecas del lenguaje.

En [14] se define el término metaprogramación como a la posibilidad de que un programa tenga
conocimiento o se manipule a sí mismo. En {\it Python} cada pequeño elemento del lenguaje
representa un objeto. A diferencia de otros lenguajes, donde la declaración de funciones y
clases no son más que punteros a direcciones de memoria donde se alojan sus respectivos códigos,
en {\it Python} cada definición tiene como resultado la instancia de una determinada clase que se
referencia a partir del nombre de dicha definición y que desde el preciso momento de su
definición dicha instancia puede ser modificada de todas las maneras que soporte el lenguaje
y la semántica del contexto.

El más popular ejemplo de lo antes expresado son los decoradores. Estos representan una de las
cualidades de más alto nivel del paradigma funcional, las funciones de orden superior (funciones
que esperan funciones como parámetros). Los decoradores ya no son una sintaxis novedosa en el
mundo de los lenguajes de programación; pero la sencillez de estos en {\it Python} sigue resaltando
por encima del resto, pues según su filosofía, los decoradores no son más que una función
simple donde el argumento es el objeto resultante de la definición subyacente. Esta es una
increíble herramienta para escribir metaprogramas pues con solo una linea de código más por
encima de una definición, ya sea de función o de clase, se puede transformar el objeto al que
apunta el nombre de la definición que cualquier otra instancia. Esto permite por ejemplo
simplificar la sintaxis para declarar una jerarquía de clases simple, donde la clase que hereda
únicamente le interesa sobreescribir un método en particular, pues bastaría con decorar la
una función "x" para que cuando se le intente ejecutar se cree la instancia clases "y"
que tiene un método "z" que llama a la función "x"

La más alta expresión de esta filosofía, donde toda definición es la instancia de un objeto,
son las metaclases. Una metaclase es el clase que describe la naturaleza de las instancias
resultantes de la declaración de nuevas clases. El lenguaje define la metaclases básica
{\bf type} y brinda las herramientas necesarias para crear nuevas y personalizar la asignación
de su respectiva metaclases para cada clase que el usuario define. El procesos de instanciación
de una nueva clase definida por el usuarios pasa por un pipeline de 3 "{\it métodos mágicos}" que
tiene origen en el método \_\_call\_\_ de la instancia de su metaclase. Esto es otra gran
característica para la flexibilización de la semántica pues por ejemplo, bajo el nombre de una
misma clase, en el momento de crear una nueva instancia, se podrían crear la instancia adecuada
de toda una jerarquía según las características de los parámetros iniciales.

Además la biblioteca estándar del lenguaje incluye el módulo {\bf inspect}, que como indica [15],
proporciona varias funciones útiles para ayudar a obtener información sobre objetos vivos
como módulos, clases, métodos y funciones. Hay cuatro tipos principales de servicios que ofrece
este módulo: comprobación de tipos, obtención del código fuente, inspección de clases y funciones,
y examen de la pila del intérprete. Lo cual supone una inmensa fuente de metadatos que unido
a todo lo antes expuestos transforman a este lenguaje en el ambiente ideal para el desarrollo de
    {\bf DSLs} y framewors de gran expresividad.

\section{Estado del Arte}

En función del marco teórico en que se desarrolló la investigación y teniendo en cuenta
que el estado del arte respecto a la descripción de espacios de búsqueda, en este momento,
se encuentra concentrado en los sistemas {\it AutoML} y bibliotecas de optimización, entonces
se realizo una selección y estudio de las herramientas del sector, que contarán con algún
mecanismo para expresar la dimension o estructura de su espacio de búsqueda. Dicha
herramienta debía ser; un mecanismo integrado con el lenguaje de propósito general
subyacente, en los que el objetivo final de cada descripción fuera la generación de
muestras. El listado final quedo integrado por:

\begin{itemize}

    \item  {\bf AutoGOAL} [4]: Biblioteca de {\it AutoML}, escrita en {\it Python}, transversal a la
          naturaleza de los problemas y de las herramientas subyacentes. Mediante su módulo {\it Grammar}
          ofrece un listados de tipos y abstracciones con las que los desarrolladores pueden describir
          su espacio de búsqueda.
    \item {\bf HyperOpt} [7]: Biblioteca de {\it Python} que intenta resolver el problema de la optimización
          paramétrica siendo independiente a la función en cuestión. La misma define una sintaxis y
          una lista de funciones para expresar la definición de cada uno de los parámetro de la
          función en cuestión
    \item {\bf Ray AI Runtime} (AIR) [8]: Ray es un marco unificado para escalar aplicaciones de {\it IA} y
              {\it Python}. Y AIR es su conjunto de herramientas de código abierto para crear aplicaciones de
              {\it IA}, en la cual incluye una lista de funciones con las que los desarrolladores pueden expresar
          las dimensiones de los distintos hiperparámetros
    \item {\bf Chocolate} [9]: Chocolate es un marco de optimización completamente asíncrono que depende
          únicamente de una base de datos para compartir información entre los trabajadores. Chocolate
          ha sido diseñado y optimizado para la optimización de hiperparámetros, donde cada evaluación
          de funciones tarda mucho en completarse y es difícil de paralelizar. Y como tal define una
          sintaxis y una lista de funciones para definir los dominós de dichos hiperparámetros
    \item {\bf Optuna} [10]: Optuna es un marco de software de optimización automática de hiperparámetros,
          especialmente diseñado para el aprendizaje automático. El cual inyecta la instancia de una
          clase predefinida para seguir la evolución de dichos hiperparámetros asi como
          una lista de funciones para describir las características de los mismos
    \item {\bf AutoGloun} [11]: Biblioteca de {\it AutoML}, escrita en {\it Python}, que permite utilizar y
          ampliar {\it AutoML} de forma sencilla, centrándose en el ensamblaje automatizado de pilas, el
          aprendizaje profundo y las aplicaciones del mundo real que abarcan datos de imágenes, textos y tablas.
          Aprovechando el ajuste automático de hiperparámetros, la selección/ensamblaje de modelos,
          la búsqueda de arquitecturas y el procesamiento de datos. Mejorar/ajustar fácilmente sus
          modelos y pipelines de datos a medida, o personalizar AutoGluon para su caso de uso. Para
          dicha personalización la biblioteca combina la definición de una sintaxis para la descripción
          estructural de los hiperparámetros con una lista de tipos para expresar la dimension de los
          mismo
    \item {\bf AutoSklearn} [12]: Biblioteca de {\it AutoML}, escrita en {\it Python}, sustentados sobre
          conjunto de herramientas de aprendizaje automático de scikit-learn. Permite a los desarrolladores
          personalizar sus modelos ofreciendo una lista de tipos con los que restringir los distintos dominios
          de cada hiperparámetro, junto con una sintaxis específica para la declaración de los mismo
    \item {\bf TPOT} [13]: TPOT es una herramienta de aprendizaje automático en {\it Python} que optimiza los
          procesos de aprendizaje automático mediante programación genética. Este define una sintaxis para
          describir la lista de modelos a explorar y sus distintos hiperparámetros
\end{itemize}

Aunque la investigación realizada en [6] se enfoca más en la clasificación de {\bf DSL} que
representan proyectos más grandes que aquellos que son objeto de estudio para esta
investigación en concreto, estudiando las clasificaciones y características que plantea,
el autor pudo seleccionar varias que son acordes para describir el estado del arte
de los DSL's que hasta el momento se dan a la tarea de describir espacios de búsqueda.
A continuación se enumeran y detallan las características con las que se pretende
describir el estado y las propiedades de los trabajos realizados en el área hasta el
momento:

\begin{itemize}
    \item {\bf Estilo de la Sintaxis Concreta}: Esta puede ser imperativa o funcional
    \item {\bf Objetivo del Sistema Subyacente}: En el caso particular del campo de interés de esta
          investigación la mayoría de las herramientas se encuentra relacionadas con el {\it ML} o
              {\it AutoML}, pero dentro de ambos campos existen múltiples subdominios y razones por las que
          sería de interés describir un dominio determinado
    \item {\bf Activo Objetivo}: Nombre con el que se describe el resultado esperado por las transformaciones
          del {\bf DSL}. En los casos analizados por [6] suelen ser archivos de textos, gráficos o llamadas
          al sistema, pero en esta investigación el autor reinterpreto esta característica, debido a la
          naturales del campo de investigación, y por tanto los activos objetivos pasan a describir a
          los efectos que provoca el empleo del mismo dentro de un programa, las instancias que
          genera o las modificaciones que espera conseguir
    \item {\bf Integración con el Lenguaje Subyacente}: Lista de herramientas y características de las
          que se valen los desarrolladores de cada uno de los {\bf DSL's} para dar lugar a los mismo dentro
          del lenguaje subyacente
    \item {\bf Desacoplamientos Descripción - Generación}: Describe el nivel de desacoplamiento entre
          las herramientas que soportan las descripciones de los distintos espacios de búsqueda con
          los mecanismos para generar las muestras de dichas descripciones. Un diseño ideal
          es aquel que permita para una misma definición probar varias formas de generar muestras.
    \item {\bf Características que Conduce el Diseño}: Detrás de las descripciones de los espacios de
          búsqueda existe mucho carga teórica de diversas esferas no solo la computación y el {\it ML},
          sino que también juegan un papel importante la estadística, las probabilidades y otros muchos
          campos de las matemáticas. Para la definición de los distintos {\bf DSL's} los autores se inspiraron
          en muchos de estos campos para dar expresividad a los mismo, dicha inspiración es la que se
          intentara reflejar cuando se resalte esta detalle para cada uno de los trabajos previos
    \item {\bf Capacidad de Generación y Definición}: En este punto se realizará una comparación según tres
          de los puntos fuertes de la solución presentada por este trabajo:
          \begin{itemize}

              \item Definición de dependencias y relaciones entre los componentes internos de una misma descripción
              \item Descripción y generación de espacio de búsqueda de dimensiones aleatorias, por ejemplo el espacio
                    de los vectores de dimension aleatoria
              \item Estructuras de controles de flujo para las descripción de espacios opcionales
          \end{itemize}

\end{itemize}


\definecolor{lavender}{rgb}{0.9, 0.9, 0.98}

\newcommand*\rot{\rotatebox{90}}
\newcommand*\OK{\ding{51}}



\begin{table}[h]
    \begin{adjustwidth}{-3cm}{-1cm}
        \centering
        \begin{tabular}{cccccccccc}
            \toprule
            \multicolumn{1}{c}{}                                                           &
            \multicolumn{1}{c}{}                                                           &
            \multicolumn{1}{c}{}                                                           &
            \multicolumn{1}{c}{}                                                           &
            \multicolumn{1}{c}{}                                                           &
            \multicolumn{1}{c}{}                                                           &
            \multicolumn{1}{c}{}                                                           &
            \multicolumn{1}{c}{}                                                           &
            \multicolumn{1}{c}{}                                                           &
            \multicolumn{1}{c}{}                                                             \\
            
            \textbf{Bibliotéca}                                                            &
            \textbf{\tabular{@{}l@{}l@{}}Estilo de                                           \\la Sintaxis\\Concreta\endtabular} &
            % \textbf{\tabular{@{}l@{}}Objetivo del\\ Sistema Subyacente\endtabular} &

            % \rot{\textbf{Activo Objetivo}}                                                            &

            \textbf{\tabular{@{}l@{}l@{}}Integración con                                     \\el Lenguaje\\Subyacente\endtabular} &

            % \begin{turn}{90}
            %     {\textbf{\tabular{@{}l@{}l@{}}Desacoplamientos \\ Descripción - \\Generación\endtabular}}
            % \end{turn} &

            \textbf{\tabular{@{}l@{}l@{}}Características                                     \\que Conduce\\el Diseño\endtabular} &
            \begin{turn}{90}
                {\textbf{\tabular{@{}l@{}}Dependencias             \\Contextuales\endtabular}}
            \end{turn} &
            \begin{turn}{90}

                {\textbf{\tabular{@{}l@{}}Dimensiones              \\Aleatorias\endtabular} }
            \end{turn}  &
            \begin{turn}{90}

                {\textbf{\tabular{@{}l@{}}Restricciones            \\Condicionales\endtabular}}
            \end{turn}   \\


            \midrule
            \rowcolor{lavender}
            AutoGOAL                                                                       &
            Funcional                                                                      &
            \tabular{@{}l@{}l@{}l@{}}
            Hace uso de descripción                                                          \\
            de tipos de los paráme-                                                          \\
            tros y resultados de las                                                         \\
            funciones de cada clase\endtabular                                             &

            \tabular{@{}l@{}l@{}l@{}l@{}}
            Orientado por tipos que                                                          \\
            hacen referencia a la                                                            \\
            función de la variables                                                          \\
            dentro del problema del                                                          \\
            AutoML\endtabular                                                              &

                                                                                           &
                                                                                           &
            \\

            HyperOpt                                                                       &
            Funcional                                                                      &
            E.1                                                                            &
            E.1                                                                            &
                                                                                           &
                                                                                           &
            \\
            \rowcolor{lavender}
            \tabular{@{}l@{}}Ray AI                                                          \\ Runtime (AIR)\endtabular                                                                                  &
            Funcional                                                                      &
            E.1                                                                            &
            E.1                                                                            &
            E.1                                                                            &
            E.1                                                                            &
            E.1                                                                              \\

            Chocolate                                                                      &
            Funcional                                                                      &
            E.1                                                                            &
            E.1                                                                            &
            E.1                                                                            &
            E.1                                                                            &
            E.1                                                                              \\


            \rowcolor{lavender}
            Optuna                                                                         &
            Imperativo                                                                     &
            E.1                                                                            &
            E.1                                                                            &
            \OK                                                                            &
            \OK                                                                            &
            \OK                                                                              \\

            AutoGloun                                                                      &
            Funcional                                                                      &
            E.1                                                                            &
            E.1                                                                            &
            E.1                                                                            &
            E.1                                                                            &
            E.1                                                                              \\

            \rowcolor{lavender}

            AutoSklearn                                                                    &
            Funcional                                                                      &
            E.1                                                                            &
            E.1                                                                            &
            E.1                                                                            &
            E.1                                                                            &
            E.1                                                                              \\


            TPOT                                                                           &
            Funcional                                                                      &
            E.1                                                                            &
            E.1                                                                            &
            E.1                                                                            &
            E.1                                                                            &
            E.1                                                                              \\
            \rowcolor{lavender}
            \tabular{@{}l@{}}Propuesta                                                       \\ de Solución\endtabular                                    &
            Funcional                                                                      &
            E.1                                                                            &
            E.1                                                                            &
            \OK                                                                            &
            \OK                                                                            &
            \OK                                                                              \\




            % row3                 & A.3                                     & B.3                                      & C.3 & D.3 & E.3 \\
            % row4                 & A.4                                     & B.4                                      & C.4 & D.4 & E.4 \\
            % \rowcolor{lavender}
            % row5                 & A.5                                     & B.5                                      & C.5 & D.5 & E.5 \\
            \bottomrule
        \end{tabular}
    \end{adjustwidth}
\end{table}

=



% \begin{table}[h]
%     \begin{adjustwidth}{-3cm}{-1cm}
%         \centering
%         \begin{tabular}{cccccccccc}
%             \toprule
%             \multicolumn{1}{c}{}                                                                      &
%             \multicolumn{1}{c}{}                                                                      &
%             \multicolumn{1}{c}{}                                                                      &
%             \multicolumn{1}{c}{}                                                                      &
%             \multicolumn{1}{c}{}                                                                      &
%             \multicolumn{1}{c}{}                                                                      &
%             \multicolumn{1}{c}{}                                                                      &
%             \multicolumn{1}{c}{}                                                                      &
%             \multicolumn{1}{c}{}                                                                      &
%             \multicolumn{1}{c}{}                                                                        \\

%             \rot{\textbf{Bibliotéca}}                                                                 &


%             \begin{turn}{90}
%                 {\textbf{\tabular{@{}l@{}}Estilo de la\\ Sintaxis Concreta\endtabular}}
%             \end{turn}                   &

%             \begin{turn}{90}
%                 {\textbf{\tabular{@{}l@{}}Objetivo del\\ Sistema Subyacente\endtabular}}
%             \end{turn}                  &

%             \rot{\textbf{Activo Objetivo}}                                                            &

%             \begin{turn}{90}
%                 {\textbf{\tabular{@{}l@{}}Integración con el\\ Lenguaje Subyacente\endtabular}}
%             \end{turn}           &

%             \begin{turn}{90}
%                 {\textbf{\tabular{@{}l@{}l@{}}Desacoplamientos \\ Descripción - \\Generación\endtabular}}
%             \end{turn} &

%             \begin{turn}{90}
%                 {\textbf{\tabular{@{}l@{}}Características que \\ Conduce el Diseño\endtabular}}
%             \end{turn}           &

%             \begin{turn}{90}
%                 {\textbf{\tabular{@{}l@{}}Dependencias  \\ Contextuales\endtabular}}
%             \end{turn}                      &

%             \begin{turn}{90}
%                 {\textbf{\tabular{@{}l@{}}Dimensiones  \\ Aleatorias\endtabular}}
%             \end{turn}                         &

%             \begin{turn}{90}
%                 {\textbf{\tabular{@{}l@{}}Restricciones  \\ Condicionales\endtabular}}
%             \end{turn}                       \\


%             \midrule
%             \rowcolor{lavender}
%             AutoGOAL                                                                                  &
%             Funcional                                                                                 &
%             AutoML                                                                                    &
%             C.1                                                                                       &
%             \tabular{@{}l@{}l@{}l@{}} Hace uso de descripción                                           \\de tipos de los parámetros y \\resultados de las funciones \\de cada clase \endtabular                               &

%             E.1                                                                                       &
%             E.1                                                                                       &
%             E.1                                                                                       &
%             E.1                                                                                       &
%             E.1                                                                                         \\

%             HyperOpt                                                                                  &
%             Funcional                                                                                 &
%             \tabular{@{}l@{}} Optimización                                                              \\Paramétrica\endtabular                               &
%             C.1                                                                                       &
%             D.1                                                                                       &
%             E.1                                                                                       &
%             E.1                                                                                       &
%             E.1                                                                                       &
%             E.1                                                                                       &
%             E.1                                                                                         \\
%             \rowcolor{lavender}
%             \tabular{@{}l@{}}Ray AI                                                                     \\ Runtime (AIR)\endtabular                                                                                  &
%             Funcional                                                                                 &
%             \tabular{@{}l@{}} Optimización                                                              \\Paramétrica\endtabular                               &
%             C.1                                                                                       &
%             D.1                                                                                       &
%             E.1                                                                                       &
%             E.1                                                                                       &
%             E.1                                                                                       &
%             E.1                                                                                       &
%             E.1                                                                                         \\

%             Chocolate                                                                                 &
%             Funcional                                                                                 &
%             \tabular{@{}l@{}} Optimización                                                              \\Paramétrica\endtabular                               &
%             C.1                                                                                       &
%             D.1                                                                                       &
%             E.1                                                                                       &
%             E.1                                                                                       &
%             E.1                                                                                       &
%             E.1                                                                                       &
%             E.1                                                                                         \\


%             \rowcolor{lavender}
%             Optuna                                                                                    &
%             Imperativo                                                                                &
%             \tabular{@{}l@{}} Optimización                                                              \\Paramétrica\endtabular       &
%             C.1                                                                                       &
%             D.1                                                                                       &
%             E.1                                                                                       &
%             E.1                                                                                       &
%             E.1                                                                                       &
%             E.1                                                                                       &
%             E.1                                                                                         \\

%             AutoGloun                                                                                 &
%             Funcional                                                                                 &
%             AutoML                                                                                    &
%             C.1                                                                                       &
%             D.1                                                                                       &
%             E.1                                                                                       &
%             E.1                                                                                       &
%             E.1                                                                                       &
%             E.1                                                                                       &
%             E.1                                                                                         \\

%             \rowcolor{lavender}

%             AutoSklearn                                                                               &
%             Funcional                                                                                 &
%             AutoML                                                                                    &
%             C.1                                                                                       &
%             D.1                                                                                       &
%             E.1                                                                                       &
%             E.1                                                                                       &
%             E.1                                                                                       &
%             E.1                                                                                       &
%             E.1                                                                                         \\


%             TPOT                                                                                      &
%             Funcional                                                                                 &
%             AutoML                                                                                    &
%             C.1                                                                                       &
%             D.1                                                                                       &
%             E.1                                                                                       &
%             E.1                                                                                       &
%             E.1                                                                                       &
%             E.1                                                                                       &
%             E.1                                                                                         \\
%             \rowcolor{lavender}
%             \tabular{@{}l@{}}Propuesta                                                                  \\ de Solución\endtabular                                    &
%             Funcional                                                                                 &
%             \tabular{@{}l@{}l@{}l@{}l@{}}Descripción                                                    \\ de Espacios \\ de Búsqueda\\ y Generación \\ de Muestras\endtabular                                                                                   &
%             C.1                                                                                       &
%             D.1                                                                                       &
%             E.1                                                                                       &
%             E.1                                                                                       &
%             E.1                                                                                       &
%             E.1                                                                                       &
%             E.1                                                                                         \\




%             % row3                 & A.3                                     & B.3                                      & C.3 & D.3 & E.3 \\
%             % row4                 & A.4                                     & B.4                                      & C.4 & D.4 & E.4 \\
%             % \rowcolor{lavender}
%             % row5                 & A.5                                     & B.5                                      & C.5 & D.5 & E.5 \\
%             \bottomrule
%         \end{tabular}
%     \end{adjustwidth}
% \end{table}









Analizando el estado del arte mediante la comparación antes expuesta se
resaltar varias puntos. La mayoría de las herramientas de la actualidad
no resuelven ninguno de los problemas a los que el presente trabajo intenta
dar respuesta. Todas las herramientas estudiadas interpretan que el mecanismo
generativo es un elemento integrado de forma natural en las descripciones de
los espacios de búsqueda, idea que el autor considera que no es el mejor
diseño con respecto a la escalabilidad y modularidad de las implementaciones.
Una cantidad considerable de los ejemplos de solución se apoyan en los diccionarios
y la comparación textual entre los nombres de los argumentos con las llaves
de dichos diccionarios o nombres que se le asignan a las distintas instancias de
las clases básicas.

Además cada uno de los {\bf DSLs} analizado presentan diseños muy influenciados
por el domino de la herramienta subyacente. En los casos en que el objetivo
principal de la misma sea la optimización paramétrica, las descripciones se
encuentran constituidas por los nombres de las distribuciones de cada una de
las variables aleatorias. Mientras que herramientas especializadas en la solución
del problema del {\it AutoML}, las herramientas descriptivas se ven más influenciada
por las funciones de estos parámetros dentro de los distintos algoritmos. Bajo este
análisis una herramienta con el propósito principal de describir los espacios de
búsqueda, como sería de propuesta de solución, debe presentar una sintaxis expresiva
respeto a los tipos que serán generados como resultado final.

Por último, se destaca el caso de la biblioteca Optuna que siendo la herramienta
que se decanta por una filosofía imperativa, cuenta con mecanismo para dar respuesta
a los problemas que dieron lugar a la presente investigación. Pese a la flexibilidad
y el potencial de la propuesta no es la sintaxis ideal para las descripciones, como se
evidencia en la elección del resto de los sistemas que se decantan por el paradigma
funcional. Las descripciones resultantes de estas sintaxis, pese la amplia gama de
dominios que puede generar, a medida que los espacios se complejizan las implementaciones
se tornan muy verbosa y relativamente poco legibles. Además esta en esta propuesta se
resolverían todas las restricciones y dependencias en tiempo de ejecución, mientras
que una sintaxis funcional da espacio a realizar múltiples optimizaciones para
minimizar el computo en tiempo de ejecución.

% \nocite{fogel2006evolutionary}