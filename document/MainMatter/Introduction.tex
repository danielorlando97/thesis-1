\chapter*{Introducción}\label{chapter:introduction}
\addcontentsline{toc}{chapter}{Introducción}

El aprendizaje automático (machine learning, {\it ML}) es uno de los más famosos
y poderosos campos de la inteligencia artificial. Sin embargo, es un campo
donde el tiempo de aprendizaje e investigación es muy superior al de desarrollo
y donde la experiencia de los investigadores juega un papel fundamental tanto
en los resultados finales como en la efectividad del procesos investigativo.

    {\it AutoML} (Automated Machine Learning), según [2], es un campo de investigación
que persigue la automatización incremental de todas las fases del desarrollo
de aplicaciones de aprendizaje automático. A diferencia del proceso de diseño
manual, {\it AutoML} permite explorar inteligentemente las mejores combinaciones
de algoritmos e hiperparámetros para la construcción de posibles soluciones.

Se han creado varias bibliotecas que aprovechan las tecnologías de {\it ML}
existentes para aplicar técnicas {\it AutoML} y asi ofrecer una forma óptima
o semi óptima de combinar dichas tecnologías para dar solución a
los problemas que puedan ser resueltos con las mismas. La mayoría de dichas
herramientas se centran en una familia específica de algoritmos ({\it como las
        redes neuronales}) o en un entorno de problema específico ({\it como el
        aprendizaje supervisado a partir de datos tabulares}). Sin embargo, en
escenarios prácticos, los investigadores necesitan combinar tecnologías de
diferentes marcos que no siempre están diseñados para interactuar entre sí.
En muchos escenarios la herramienta ideal de {\it AutoML} es aquella que sea
transversal a la naturaleza del problema, flexible para combinar herramientas
en principio incompatibles y extensible para agregar nuevas tecnologías o
implementaciones propias. Estas características describen y definen a
    {\it Automatic Generation, Optimization And Artificial Learning} ({\bf AutoGOAL}).

    {\bf AutoGOAL} se autodefine en su documentación oficial[4] como:

\begin{verbatim}
    " ...Una biblioteca de Python para encontrar automáticamente la me-
    jor manera de resolver una tarea determinada.... 
    Técnicamente hablando, AutoGOAL es un marco para la síntesis de pro-
    gramas, es decir, encontrar el mejor programa para resolver un pro-
    blema dado, siempre que el usuario pueda describir el espacio de to-
    dos los programas posibles"
\end{verbatim}


Para que el usuario pueda describir el espacio de todos los programas posibles,
la biblioteca se apoya en un de sus submódulos principales, {\it Grammar}. El mismo
proporciona una abstracción de lo que se define como un algoritmo y un conjunto
de tipos (números, booleanos, etc.), para poder definir la naturaleza de los
valores de entrada, salida e hiperparámetros de los algoritmos que formen parte
del espacio de interés del usuario.

Con la lista de descripciones del usuario se infiere una gramática libre
del contexto que describe el lenguaje de todos los potenciales programas validos
que se pueden construir combinando los algoritmos aportados. Al momento de inferir
la gramática se analizan los tipos de entrada y salida de todos los algoritmos y se
excluyen los programas donde se intente secuenciar dos implementaciones tales que
la salida de una no sea consistente con la entrada de la siguiente. De esta
manera se logra descarta una gran cantidad de secuencias invalidas, incluso
antes de empezar a buscar, pero como se puede ver en los resultados de [1]
todavía quedan algunas {\it instancias invalidas}.

En muchos casos las {\it instancias invalidas} son consecuencias de los
parámetros iniciales de la búsqueda, como el tiempo de espera o la capacidad de memoria.
En estos casos durante la ejecución de la instancia en cuestión se superan los límites
prefijados y automáticamente es descartada como posible solución. Como para la
biblioteca un algoritmo es una función de la cual solo conoce sus tipos de entrada y
salida, prever el fallo de las instancias antes descritas previo a su ejecución,
es equivalente al {\it Halting Problem}, problema que {\it Alan Turing} demostró en 1936
que es indecidible en una máquinas de Turing [3].

Sin embargo existe otro conjunto de {\it instancias invalidas} relacionadas a la validez
y consistencia de los hiperparámtros iniciales. A lo largo del proceso de búsqueda
se exploran las "posibles soluciones" a partir de las distintas instancias generadas
por la gramática que se infirió de las definiciones del usuario. Como la misma
es libre del contexto los valores que se generan para cada símbolo terminal de la misma
son independientes entre si, pero, existen modelos en los que por definición o por
experiencia práctica se presenta una cierta dependencia entre los valores de sus
hiperparámetros.

Para describir dicho modelo en el contexto de {\bf AutoGOAL} se presentan dos opciones;
se puede definir el modelo tal cual, asignar a cada hiperparámetro la descripción
más extensa de su dominio y en el interior del algoritmo controlar que los mismos
cumplan con dichas reglas contextuales y en casos contrario lanzar un excepción en
tiempo de ejecución. O por el contrario, teniendo en cuanta dichas dependencias,
que el usuario defina un algoritmo por cada combinación de subdominios compatibles
de los hiperparámetros.

En la práctica, debido a las limitaciones de la biblioteca para describir dichas
dependencias contextuales, los desarrolladores prefieren incluir dichas {\it instancias
        invalidas} a su espacio de búsqueda inicial, pues el procesos de evitarlas suele
ser bastante tediosos y poco escalable. Provocando consigo una dilatación del
tiempo de ejecución de {\bf AutoGOAL}.

Dichas limitaciones dieron lugar a que los autores de la biblioteca propusieran
nuevo problema a resolver, el desarrollo de una herramienta capaz de describir de
la forma más expresiva y simple posible la estructura interna de los
distintos espacios de búsqueda, las dependencias y relaciones existentes entre sus
componentes, y con la capacidad de generar muestras, dada una descripción
previa, de forma tal que cada uno de los dominio internos se reajuste al contexto
específico del procesos generativo en cuestión.

En respuesta a este nuevo problema el presente documento plantea el desarrollo de una
nueva biblioteca que cuente con todas las arquitecturas y herramientas necesarias para
crear un {\bf DSL} capaz de describir los distintos espacios de búsquedas bajo una
filosofía "{\it de abajo hacia arriba}" ({\it Bottom-Up}), mediante el cual apoyado
en la definición de algunos tipos básicos el usuario pueda ser capaz de componer la
estructura interna de su espacio de interés. Dicha herramienta cuenta además con una
sintaxis, inspirado en el paradigma funcional, para declararles restricciones y
relaciones a cada uno de los distintos subespacios, declaraciones que dan lugar a la
definición de varios {\bf AST's} ({\it Astract Syntaxis Trees, Árboles de Sintaxis
        Abstracta}) los cuales son visitados al momento del muestreo, para acotar los distintos
subdominos y validar cada uno de las selecciones internas.

Toda esta investigación y desarrollo se realizó con el objetivo de crear una
herramienta con la que se pueda describir detalladamente, en un lenguaje de alto nivel,
los distintos espacios de búsqueda. Descripciones que debían ser,
por las características de la biblioteca, escalables, mantenibles, expresivas,
independientes de los procesos y algoritmos de generación de muestras, pero a
su vez capaz de transmitirle a estos los distintos dominios dinámicos para
cada contextos en cuestión.

Luego en un plano más general se esperaba darle respuesta a las limitaciones
de {\bf AutoGOAL} que dieron lugar al problema inicial y que el resultado final sea
de utilidad en todos aquellos escenarios donde sea de interés describir espacios
aleatorios y generar muestras del mismo, como pueden ser los algoritmos genéticos,
donde puede ser interesante que la descripción de la población sea lo más expresiva
posible.

El documento esta organizado en tres capítulos. Un primer capítulo donde se analiza
el marco teórico en que se realizaron las implementaciones y los trabajos realizados
en el sector hasta la fecha. Otro donde se detalla la propuesta de soluciones, sus
objetivos, alcance e implementación. Y por último un capítulo donde apoyado en ejemplos
se evidenciara la efectividad y expresividad de la propuesta.
