\chapter*{Introducción}\label{chapter:introduction}
\addcontentsline{toc}{chapter}{Introducción}


El aprendizaje automático (machine learning, {\it ML}) es uno de los más famosos
y poderosos campos de la inteligencia artificial. Sin embargo, es un campo
donde el tiempo de aprendizaje e investigación es muy superior al de desarrollo
y donde la experiencia de los investigadores juega un papel fundamental tanto
en los resultados finales como en la efectividad del proceso investigativo.

    {\it AutoML} (Automated Machine Learning), según (\cite{autoML}), es un campo de investigación
que persigue la automatización incremental de todas las fases del desarrollo
de aplicaciones de aprendizaje automático. A diferencia del proceso de diseño
manual, {\it AutoML} permite explorar inteligentemente las mejores combinaciones
de algoritmos e hiperparámetros para la construcción de posibles soluciones.

Se han creado varias bibliotecas que aprovechan las tecnologías de {\it ML}
existentes para aplicar técnicas {\it AutoML} y de esta manera ofrecer una forma óptima
o semi óptima de combinar dichas tecnologías para dar solución a
los problemas que puedan ser resueltos con las mismas. La mayoría de dichas
herramientas se centran en una familia específica de algoritmos ({\it como las
        redes neuronales}) o en un entorno de problema específico ({\it como el
        aprendizaje supervisado a partir de datos tabulares}). Sin embargo, en
escenarios prácticos, los investigadores necesitan combinar tecnologías de
diferentes marcos que no siempre están diseñados para interactuar entre sí.
En muchos escenarios la herramienta ideal de {\it AutoML} es aquella que sea
transversal a la naturaleza del problema, flexible para combinar herramientas
en principio incompatibles y extensible para agregar nuevas tecnologías o
implementaciones propias. Estas características describen y definen a
    {\it Automatic Generation, Optimization And Artificial Learning} ({\bf AutoGOAL}).

    {\bf AutoGOAL} se autodefine en su documentación oficial (\cite{autogoal}  como:

\begin{verbatim}
    `` ...Una biblioteca de Python para encontrar automáticamente la me-
    jor manera de resolver una tarea determinada.... 
    Técnicamente hablando, AutoGOAL es un marco para la síntesis de pro-
    gramas, es decir, encontrar el mejor programa para resolver un pro-
    blema dado, siempre que el usuario pueda describir el espacio de to-
    dos los programas posibles''
\end{verbatim}


Para que el usuario pueda describir el espacio de todos los programas posibles,
la biblioteca se apoya en un de sus submódulos principales, {\it Grammar}. El mismo
proporciona una abstracción de lo que se define como un algoritmo y un conjunto
de tipos (números, booleanos, etc.), para poder definir la naturaleza de los
valores de entrada, salida e hiperparámetros de los algoritmos que formen parte
del espacio de interés del usuario.

Con la lista de descripciones del usuario se infiere una gramática libre
del contexto que describe el lenguaje de todos los potenciales programas válidos
que se pueden construir combinando los algoritmos aportados. Al momento de inferir
la gramática se analizan los tipos de entrada y salida de todos los algoritmos y se
excluyen los programas donde se intente secuenciar dos implementaciones tales que
la salida de uno no sea consistente con la entrada del siguiente. De esta
manera se logra descartar una gran cantidad de secuencias inválidas, incluso
antes de empezar a buscar, pero como se puede ver en los resultados de (\cite{estevez-velarde-etal-2019-automl})
todavía quedan algunas {\it instancias inválidas}.

\section*{Motivación}

En muchos casos las {\it instancias inválidas} de una ejecuciones de {\bf AutoGOAL} son consecuencias de los
parámetros iniciales de la búsqueda, como el tiempo de espera o la capacidad de memoria.
En estos casos, durante la ejecución de la instancia en cuestión se superan los límites
prefijados y es automáticamente descartada como posible solución. Como para la
biblioteca un algoritmo es una función de la cual solo conoce sus tipos de entrada y
tipo de salida, prever el fallo de las instancias antes descritas previo a su ejecución,
es equivalente al {\it Halting Problem}, problema que {\it Alan Turing} demostró en 1936
que es indecidible en una máquina de Turing (\cite{Hopcroft+Ullman/79/Introduction}).

Sin embargo, existe otro conjunto de {\it instancias inválidas} relacionadas a la validez
y consistencia de los hiperparámtros iniciales. A lo largo del proceso de búsqueda
se exploran las ``posibles soluciones'' a partir de las distintas instancias generadas
por la gramática que se infirió de las definiciones del usuario. Como la misma
es libre del contexto, los valores que se generan para cada símbolo terminal de la misma
son independientes entre sí, pero, existen modelos en los que por definición o por
experiencia práctica se presenta una cierta dependencia entre los valores de sus
hiperparámetros.

Para describir dicho modelo en el contexto de {\bf AutoGOAL} se presentan dos opciones.
La primera sería definir el modelo de la forma más simple posible, asignando a cada
hiperparámetro la descripción más extensa de su dominio y en el interior del algoritmo
controlar que los mismos cumplan con dichas reglas contextuales y en caso contrario
lanzar una excepción en tiempo de ejecución. Por otro lado, la segunda opción sería, teniendo
en cuanta dichas dependencias, que el usuario defina un algoritmo por cada combinación de
subdominios compatibles de los hiperparámetros.

\section*{Problema}

En la práctica, debido a las limitaciones de la biblioteca para describir
restricciones y dependencias contextuales, los desarrolladores prefieren incluir
en sus espacios de búsqueda iniciales los elementos que no cumplen  
con dichas reglas contextuales, pues el proceso de evitarlas suele
ser bastante tediosos y poco escalable. Provocando consigo una dilatación del
tiempo de ejecución de {\bf AutoGOAL}.

Dichas limitaciones dieron lugar a que los autores de la biblioteca propusieran
un nuevo problema a resolver, el desarrollo de una herramienta capaz de describir de
la forma más expresiva y simple posible la estructura interna de los
distintos espacios de búsqueda, las dependencias y relaciones existentes entre sus
componentes, y con la capacidad de generar muestras, dada una descripción
previa, de forma tal que cada uno de los dominios internos se reajuste al contexto
específico del proceso generativo en cuestión.

\section*{Hipótesis}

En respuesta al problema planteado, la investigación realizada plantea el desarrollo de una
nueva biblioteca que cuente con todas las arquitecturas y herramientas necesarias para
crear un {\bf DSL} capaz de describir los distintos espacios de búsquedas bajo una
filosofía ``{\it de abajo hacia arriba}'' ({\it Bottom-Up}), mediante el cual apoyado
en la definición de algunos tipos básicos el usuario pueda ser capaz de describir la
estructura interna de su espacio de interés. Dicha herramienta cuenta además con una
sintaxis, inspirada en el paradigma funcional, para declarar restricciones y
relaciones en cada uno de los distintos subespacios. Estas declaraciones dan lugar a la
definición de varios {\bf AST's} ({\it Astract Syntaxis Trees, Árboles de Sintaxis
        Abstracta}) los cuales son visitados al momento del muestreo para acotar los distintos
subdominios y validar cada uno de las selecciones internas.

\section*{Objetivos}

La presente investigación y el desarrollo de su propuesta de solución se realizaron con el objetivo de crear una 
herramienta con la que se pueda describir detalladamente, en un lenguaje de 
alto nivel, los distintos espacios de búsqueda. Con dicha herramienta se 
pretendía resolver las limitaciones de {\bf AutoGOAL} que dieron lugar al problema 
inicial. Además, se esperaba que dicha herramienta fuera de utilidad en todos 
aquellos escenarios donde existiera interés en describir espacios aleatorios 
y generar muestras del mismo, como pueden ser los algoritmos genéticos, donde 
sería interesante que la descripción de la población sea lo más expresiva 
posible.

Para llevar a cabo dicha investigación y desarrollo se señalaron una serie 
de objetivos específicos relacionados con cada una de las fases de ambos 
procesos. Los mismos se detallan a continuación:
\begin{itemize}
    \item Estudio de los principales {\bf DSL}s y frameworks que tuvieran como objetivo 
    principal la descripción de espacios de búsqueda. Con dicho estudio se 
    pretendía detectar las principales limitaciones expresivas y los patrones 
    de diseño más comunes.
    \item Diseñar una sintaxis declarativa y expresiva con la cual se pudiera describir 
    las dimensiones, estructuras y restricciones de los espacios de búsqueda. 
    Dicha sintaxis debía presentar mecanismos para resolver las limitaciones 
    detectadas. Principalmente, debía poder representar dependencias contextuales, 
    limitación que dio origine a la presente investigación.
    \item Desarrollar una jerarquía de clases que definiera un {\bf DSL} que representara la 
    sintaxis diseñada anteriormente. Dicho {\bf DSL} debía ser capaz de integrarse con la 
    biblioteca {\bf AutoGOAL}.\
    \item Desarrollar un mecanismo generador que fuese capaz de generar muestras 
    aleatorias dada una descripción implementada con el {\bf DSL} anteriormente 
    definido. Dicho mecanismo debía ser extensible y modular, para que los 
    usuarios finales de la herramienta pudieran definir sus propios protocolos 
    y distribuciones para la generación aleatoria sin modificar las descripciones 
    realizadas.
    \item Someter la implementación concreta de la propuesta de solución a una fase 
    experimental para obtener resultados cuantitativos y cualitativos sobre las 
    capacidades expresivas y generativas de la propuesta.
    \item Realizar un análisis basado en los resultados experimentales, 
    cuestionando la validez y utilidad tanto de la implementación 
    concreta como de la investigación en general. 
\end{itemize}



\section*{Estructura del Documento}

El presente documento está organizado en tres capítulos. Un primer capítulo donde se analiza
el marco teórico en que se realizaron las implementaciones y los trabajos realizados
en el sector hasta la fecha. Un segundo capítulo donde se detalla la propuesta de soluciones, sus
objetivos, alcance e implementación. Y un último un capítulo donde apoyado en ejemplos
se mide la efectividad y expresividad de la propuesta.
