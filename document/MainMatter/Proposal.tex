\chapter{Propuesta y Detalles de Implementación}\label{chapter:proposal}

Las limitaciones expresivas no solo se encuentra en los sistemas analizados
en capítulos anteriores. El desarrollo de soluciones para problemas de búsqueda,
además de una estrategia de búsqueda y un método de evaluación, usualmente cuenta
con la implementación de mecanismos imperativos y poco descriptivos para generar
muestras del espacio de búsqueda. Para hacer frente a estas limitaciones expresivas
el presente documento plante le uso de un {\bf DSL} implementado para que el análisis de
los problemas de búsqueda partan de una descripción expresiva y de alto nivel de su
espacio de búsqueda.

Dicha herramienta fue desarrollada pensando en la comodidad del usuario a la hora de
describir y construir su espacio de interés. Por dicha razón, el biblioteca plantea
una filosofía constructiva, {\it “de abajo hacia arriba”} ({ \it “Bottom-Up”}), para la
construcción de dichas descripciones. La misma define una serie de patrones de diseño
con los que se puede expresar el espacio de búsqueda relativo una clase de {\bf Python} en
concreto, partiendo de los tipos básicos del lenguaje ({\it int},{\it float},{\it bool},
{\it str}).

Con dichos patrones el usuario puede describir la estructura y dimensión del espacio
relativo a la clase en cuestión, así como las características de cada uno de sus
componentes internos. Además como el origen de la presente investigación se encuentra
en la imposibilidad de {\bf AutoGOAL} para describir dependencias contextuales, dichas pautas
fueron ideadas para que se pudieran expresar también dependencias y relaciones entre los
componentes internos de la clase en cuestión.

La sintaxis que propone el {\bf DSL} para describir los subdominios internos, en su mayoría,
son expresiones con estilo funcional y declarativo. La intención del autor era intentar,
en la medida de lo posible, que las descripciones fueran lo más parecidas a las definiciones
de dominios matemáticos. En las definiciones matemáticas se suele expresar las características
de un conjunto a partir de las características singulares de un elemento cualquiera de este.
Un ejemplo simple del modelo al que se aspiraba es la siguiente definición de un subconjunto de
los números naturales:

\newcommand\Set[2]{\{\,#1\mid#2\,\}}
\newcommand\SET[2]{\Set{#1}{\text{#2}}}

\begin{align*}
    A := \Set{x \in \mathbb{N}}{x \ge 100, x \neq 101} \\
\end{align*}

Por otra parte, la propuesta de solución planteada, aporta sus propios mecanismos para la
generación de muestras de cada espacio en particular. Dichos mecanismos tiene como base la
generación de variables aleatorias y aunque la biblioteca cuenta con la implementación de
algunas de las distribuciones básicas, el usuario puede definir las suyas propias e incluso
redefinir las ya existentes. Asumiendo la existencia de dichas implementaciones, el resto del
procesos generativo se centra en la resolución de cada uno de los dominios concretos para cada
uno de los subespacios. Dichos dominios se encuentra expresados con la sintaxis del {\bf DSL}
y puede contener dependencias entre ellos. Entre la listas de restricciones que el usuario puede
describir es importante resaltar dos de ellas, que supone el mayor reto durante el proceso de
inferencia de los distintos dominios.

Las dependencias contextuales son un de los logros más importantes del presente trabajo. Para
su resolución el mecanismo generativo necesita reconocer el orden topológico del grafo de
restricciones, para generar cada uno de los miembros en orden ascendente según grado de
dependencia. Como a medida que se definen los distintos dominios dinámicos se van generando
muestras de los distintos subespacios entonces el mecanismo generador necesita una última fase
en la que se revise la coherencia entre la muestra y las restricciones descritas.

El {\bf DSL} propuesto plantea también una sintaxis de restricciones definidas por funciones
externas, las cuales en muchos casos el mecanismo generativo las considera funciones
de {\it "caja negra”}. En estos casos, al igual que sucede con las dependencias contextuales, es
necesario un análisis posterior al procesos generativo. Dicho análisis en esta ocasión es
mucho más propenso a errores, pues el mecanismo generativo no cuenta con información suficiente
para incluir a estas funciones en el proceso de inferencia y redefinición de cada domino en
concreto. La filosofía {\it “prueba y error”} a la hora de generar muestras aleatorias es el
mecanismo más simple posible para crear elementos que cumple con un cantidad finita de condiciones.
Aunque el autor ideo todos los protocolos que pudo para que la creación de muestras fuera lo más
lineal posible, para el casos de las funciones de {\it "caja negra”} lo más que se puede hacer es
optimizar el proceso de {\it “prueba y error”} aprendiendo de dichas equivocaciones.


\section{Descripción De La Biblioteca}

La cara visible de la biblioteca propuesta es la clase {\bf Domain} la cual es una factoría de espacios
aleatorios. Dicha clase, los tipos básicos y las funciones {\it lambdas} de {\bf Python} son todas
las herramientas necesarias para que los usuarios puedan definir sus clases, decorarlas describiendo
sus espacios búsqueda y posteriormente generar instancias de las mismas.


\begin{listing}[!ht]
    \begin{minted}{Python}
        from search_space import Domain
        ADomain = Domain[int]()
        a, _ = ADomain.get_sample()
    \end{minted}
    \caption{Ejemplo básico}
    \label{lst:example}
\end{listing}





La sintaxis del examplo \ref{lst:example} es la mínima expresión del {\bf DSL} propuesto, donde basta con
definir el tipo del espacio de búsqueda para generar muestras dentro de la versión más extensa dicho
espacio. Y aunque la herramienta se desarrollo con la intención de describir espacios más complejos,
como se puede ver en el ejemplo se puede usar para generar muestra de los tipos básicos. Dicha
funcionalidad "segundaría" en realidad es la base sobre la que se sustenta el resto de funciones de la
biblioteca. Y como tal el autor trabajo para que fueran lo suficientemente eficiente, nunca será igual
de rápido que los métodos nativos del lenguaje pero en futuros capítulos se evidenciará que la diferencia
no es extrema.

\begin{listing}[!ht]
    \begin{minted}{Python}
        from search_space import Domain
        ADomain = Domain[int] | (lambda x: (x >= 100, x != 101))
        a, _ = ADomain.get_sample()
    \end{minted}
    \caption{Ejemplo básico}
    \label{ex:func}
\end{listing}


En el ejemplo \ref{ex:func} se muestra uno de los mecanismos más importantes definido por el
    {\bf DSL}, la sintaxis para declarar restricciones. Con esta herramienta el usuario puede ser capaz de
expresar las dimensiones y características de sus espacios de interés. Como en los problemas de búsqueda el
costo temporal suele ser directamente proporcional a las magnitudes del espacio de búsqueda entonces es de
vital importancia que al describir el espacio de búsqueda de un problema de búsqueda se puede ser lo más
concreto posible, de forma tal que en el espacio inicial no se incluyan elementos que no representa soluciones
factibles. El autor espera que se note las similitudes entre esta sintaxis y las definiciones matemáticas, en
concreto las de este ejemplo con el subconjunto antes definido.


\begin{listing}[!ht]
    \begin{minted}{Python}
        from search_space import Domain

        class MySeachSpace:
            def __init__(
                self, 
                a: int = Domain[int](min = 10, max = 20)
                b: float = Domain[float] | (lambdas x: (x != 0.5, x < 1))
                c: str = Domain[str](options = ["SearchSpace", "Domain"])
            )
                self.a, self.b, self.c = a, b, c

        MySeachSpaceDomain = Domain[MySeachSpace]()
        my_seach_space_instance, _ = MySeachSpaceDomain.get_sample()
    \end{minted}
    \caption{Ejemplo básico}
    \label{ex:class}
\end{listing}





Como se ve en el ejemplo \ref{ex:class} número tres la filosofía de la herramienta es que el usuario puede “decorar” sus
clases declarando los distintos subespacios que la misma contiene. A partir de dicha implementación, basta
con replicar la sintaxis del primer o del segundo ejemplo, cambiar el tipo de {\bf Domain}, y las restricciones
en caso de ser necesario, para generar una instancia aleatoria del nuevo espacio descrito. Que la clase
    {\bf Domain} no solo soporte los tipos básicos sino que también pueda fabricar espacios de clases definidas por
los usuarios es la característica que hace que la propuesta de solución sea extremadamente extensible. Las
descripciones del usuario puede ser tan anidadas como se necesite, mientras que las bases de la jerarquía
planteada en cada nivel estén definidas y debidamente descritas.



\begin{listing}[!ht]
    \begin{minted}[autogobble]{Python}
        from autogoal.grammar import ContinuousValue, CategoricalValue, DiscreteValue

        class MySeachSpace:
            def __init__(
                self, 
                a: DiscreteValue(10, 20)
                b: ContinuousValue(0, 1)
                c: CategoricalValue("SearchSpace", "Domain")
            )
                if b in [0.5, 1]:
                    raise ValueError('Invalid hyperparams')
                self.a, self.b, self.c = a, b, c

        MySeachSpaceDomain = Domain[MySeachSpace]()
        my_seach_space_instance, _ = MySeachSpaceDomain.get_sample()
    \end{minted}
    \caption{Ejemplo básico}
    \label{ex:autogoal}
\end{listing}


El diseño final del {\bf DSL} planteado para la decoración de las clases se definió principalmente con la idea de
que los espacios sencillos se pudieran expresar de forma sencillas, y luego aquellos con estructuras más
complejas tuviera descripciones expresivas pero no necesariamente sencillas. Haciendo referencia una vez
más a los origenes de este trabajo, se puede comparar los ejemplos \ref{ex:autogoal} y \ref{ex:class} donde se
evidencia la diferencia entre una clase decorada con las herramientas de {\bf AutoGOAL} y otra con la propuesta
presentada. Notes que efectivamente son espacios combinados pero extremadamente simples. En ambos casos las
modificaciones a lo que seria una clase de {\bf Python} común son mínimas. Nótese además que el {\bf DSL}
planteado es un tanto más verbosos que la sintaxis de {\bf AutoGOAL}; pero esto se debe a la propuesta de solución
aboga por una sintaxis coherente y consistente con el {\it “tipado estático”} de {\bf Python}, mientras que
    {\bf AutoGOAL} se apoya en las bondades del tipado dinámico.

Como se comento anteriormente, dos de los aportes más importante de la propuesta son las dependencias
contextuales y las descripciones de “caja negra”. Ambas funcionalidades son alteraciones de la sintaxis
antes expuesta para expresar las restricciones de un domino especifico. En el ejemplo numero 5 se muestra
uno de los casos más clásicos de dependencias contextuales, el espacio aleatoria de dos números donde el
primero es mayor que el segundo. En dicho ejemplo se escribe la función de restricciones como una función
de dos parámetros donde el primero es un elemento cualquiera del espacio en cuestión y el segundo es un
elemento cualquiera de una definición previa. Dado estos dos elementos cuales quiera, el usuario puede
describir una relación entre ambos espacios a partir de la relación existente entre dos de sus elementos.

Es probable que el lector, al analizar el caso de estudio seleccionado para mostrar la sintaxis de dependencias
contextuales, piense en que dicho problema se describe mejor con la definición matemática anterior que con la
clase del ejemplo 5. Y es que la intención del ejemplo 5 era únicamente introducir la sintaxis para la decoración
de las clases con espacios sensibles al contexto. El DSL cuenta su propia sintaxis para definir espacios de más
de una dimensión, como seria el caso del dominio planteado en la definición matemática anterior. Dicha sintaxis
no solo se limita a dimensiones predefinidas, ni a restricciones de indices fijos como es el caso del ejemplo 6
que intenta simular la definición teórica del problema planteado como ejemplo. Como se puede ver en el ejemplo 7,
la sintaxis definida para los espacios multidimensionales es capaz de describir dominios donde el tamaño de las
dimensiones esta descrito por otro espacio aleatorio y las restricciones se les aplican secuencialmente a cada
subelemento de cada elemento de dicho dominio.

Nótese que, en estos últimos ejemplo se evidencia la flexibilidad de la sintaxis para definirle restricciones a un
nuevo domino. El número de variables de entrada del delegado de restricciones puede crecer según las necesidades
del usuario; siendo siempre mayor que uno, pues la primera variable siempre hace referencia a un elemento cualquier
del domino en cuestión. Salvo dicho primer elemento, todas las variables de la función de restricciones a las que
el usuario no le asigne un dominio externo harán referencia a indices dinámicos que apuntan a cada una de las
dimensiones del espacio en cuestión de forma ascendente. Por ejemplo, en la definición del parámetro fulano las
variables x, i, j, y k hacen referencia a un elemento cualquiera dominio de fulano, al indice dinámico de la
dimensión cero de dicho elemento, al indice dinámico de la dimensión uno de dicho elemento y a un elemento
cualquiera del dominio de mengano.

El caso de las restricciones de “caja negra” se ejemplifica con el ejemplo número 8. Con esta funcionalidad
la biblioteca le deja la puerta abierta al usuario para que pueda describir su dominio de forma imperativa.
Como se puede ver en el ejemplo; solo es necesario decorar una función de Python donde se describa un
procesamiento sobre los parámetros de entrada, para que el usuario pueda incluir la llamada a esta dentro de
la lista de restricciones. El DSL incluirá las llamadas a dicha función dentro de su mecanismo generativo.
Luego de que culmine el computo de la misma, dicho mecanismo utilizará sus resultados para realizar el resto
de operaciones que fueron descrita en la lista de restricciones. Que la ejecución de estas funciones de
“caja negra” se realicen antes o después del procesos de inferencia y redefinición de los distintos dominios
dinámicos en cuestión depende únicamente de los parámetros con los que el usuario llame a dicha función dentro
de la lista de restricciones. El resultado de esta función se tendrá en cuenta para al redefinición del dominio
en cuestión siempre que sus parámetros no hagan referencia al dueño de dicho dominio, osea una función de
"caja negra" no puede modificar el domino de una variable aleatoria de la que depende su resultado. En
cualquier otro caso, las funciones de “caja negra” y todas sus dependencias se resuelven previo a la
redefinición del domino en cuestión.

Para terminar con la descripción de la sintaxis del DSL propuesto y siendo consecuente con la comparación realizada
en el estado del arte, luego de haber expuesto tanto su mecanismo para describir dependencias contextuales como su
habilidad para expresar espacios de dimensiones dinámicas, véase en el ejemplo numero 10 la sintaxis propuestas para
expresar descripciones de espacios con restricciones condicionales. El DSL no aporta ninguna estructura de control
de flujo para describir estas restricciones condicionales, este anima al usuario a plantear dichas restricciones
haciendo uso de operadores lógicos. Dichos operadores presentan una implementación de corte, osea de si suficiente
con el resultado del operador de la izquierda para determinar el resultado final de la operación entonces el operador
de la derecha nunca será computado. El autor considera dicha propuesta como la opción más expresiva analizada; aunque
lo ideal seria poder incluir estructura de control de flujo al estilo de las compresión de listas  (“list comprehension”)
de Python, pero la flexibilidad de dicho lenguaje no alcanza ese punto.

Como se ha comentado con anterioridad la propuesta de solución debe plantear dos mecanismos, uno para describir espacios
de búsquedas y otro para generar muestras de lo previamente descrito. Hasta el momento se ha hecho énfasis en el componente
descriptivo y a lo largo de toda las explicaciones anteriores se ha hecho referencia a distintos procesos del componente generativo.

Dicho componente, desde sus inicios, se desarrollo con la intención de alejarse lo más posible de las filosofías de
“prueba y error”. Este, partiendo de la base de un espacio de búsqueda descrito con las herramientas anteriormente
detalladas, modela todas las descripciones como una jerarquía de clases sobre la cual pueda detectar las distintas
relaciones entre los distintos espacios, reconocer el orden topológico de dichas descripciones y preparar todas las
herramientas necesarias para construir las muestras del espacio en cuestión. En el momento que una instancia de la
clase Domain llaman al método get\_sample el mecanismo generativo, apoyado en la jerarquía antes descrita, comienza
a construir la nueva muestra en orden topológico y con la misma filosofía de las descripciones “de abajo hacia arriba”.

Como se explicó anteriormente todo el procesos generativo esta sustentado por la generación de variables aleatorias básicas
con una cierta distribución. Por esta razón, los procesos realmente aleatorios dentro de la creación de muestras es la
selección de opciones, como es el casos de los tipos básicos string y bool, y la generación numérica, para los int y los
float. El resto del mecanismo consiste en crear las instancias de los tipos adecuados con los parámetros correctos. Para que
la generación sea un proceso lineal el componente generativo trasmite cada una de las restricciones descritas en los niveles
superiores hacia los niveles inferiores donde se encuentran los tipos básicos antes mencionados. Con la interpretación de
estas restricciones la biblioteca le puede asignar a cada uno de los tipos básicos el domino más amplio que cumpla con todas
las condiciones para que el resultado del procesos aleatorio sea consecuente y efectivo.

La coherencia y consistencia de las muestras con sus respectivas descripciones es uno de los temas más importantes de los que
se debe encargar el componente generativo, sobre todo con la presencia de dependencias contextuales o restricciones de
“caja negra”. Con respecto al primer caso, como se a mencionado en diversas ocasiones, la generación de muestras se realiza
de forma constructiva en orden topológico; por lo que al momento de generar muestras de un subcomponente especifico ya se ha
generado las muestras de las que depende dichos subcomponentes. Del razonamiento anterior se deriva que al momento de generar
un valor aleatorio para cualquiera de los tipos básicos dentro del su dominio dinámicamente inferido ya se encuentran resueltas
las restricciones contextuales.


\begin{algorithm}[H]
    \SetAlgoLined
    \KwData{this text}
    \KwResult{how to write algorithm with \LaTeX2e }
    initialization\;
    \While{not at end of this document}{
        read current\;
        \eIf{understand}{
            go to next section\;
            current section becomes this one\;
        }{
            go back to the beginning of current section\;
        } }
    \caption{How to write algorithms}
\end{algorithm}



% \begin{longtable}{|p{2.8cm}|p{0.6cm}|p{1.6cm}|p{2.2cm}|P{0.8cm}|p{2.6cm}|p{2.5cm}|}
%     \caption{My data}
%     \label{tab:table3}                                                                                                           \\
%     \textbf{Name} & \textbf{Year} & \textbf{ID} & \textbf{Address} & \textbf{Salary} & \textbf{Skills} & \textbf{Qualifications} \\
%     \hline
%     Some text     & some text     & some text   & 5                & some text       & Som text        &                         \\
%     \hline
%     Some text     & some text     & some text   & 5                & some text       & Som text        &                         \\
%     \hline
%     Some text     & some text     & some text   & 5                & some text       & Som text        &                         \\
%     \hline
%     Some text     & some text     & some text   & 5                & some text       & Som text        &                         \\
%     \hline
%     Some text     & some text     & some text   & 5                & some text       & Som text        &                         \\
%     \hline
%     Some text     & some text     & some text   & 5                & some text       & Som text        &                         \\
%     \hline
%     Some text     & some text     & some text   & 5                & some text       & Som text        &                         \\
%     \hline
%     Some text     & some text     & some text   & 5                & some text       & Som text        &                         \\
%     \hline
%     Some text     & some text     & some text   & 5                & some text       & Som text        &                         \\
%     \hline
% \end{longtable}


% \begin{center}
% \begin{longtable}{|l|l|l|}
%     \caption[Feasible triples for a highly variable Grid]{Feasible triples for
%     highly variable Grid, MLMMH.} \label{grid_mlmmh}                                                                                                    \\

%     \hline \multicolumn{1}{|c|}{\textbf{Time (s)}} & \multicolumn{1}{c|}{\textbf{Triple chosen}} & \multicolumn{1}{c|}{\textbf{Other feasible triples}} \\ \hline
%     \endfirsthead

%     \multicolumn{3}{c}%
%     {{\bfseries \tablename\ \thetable{} -- continued from previous page}}                                                                               \\
%     \hline \multicolumn{1}{|c|}{\textbf{Time (s)}} &
%     \multicolumn{1}{c|}{\textbf{Triple chosen}}    &
%     \multicolumn{1}{c|}{\textbf{Other feasible triples}}                                                                                                \\ \hline
%     \endhead

%     \hline \multicolumn{3}{|r|}{{Continued on next page}}                                                                                               \\ \hline
%     \endfoot

%     \hline \hline
%     \endlastfoot

%     0                                              & (1, 11, 13725)                              & (1, 12, 10980), (1, 13, 8235), (2, 2, 0), (3, 1, 0)  \\
%     2745                                           & (1, 12, 10980)                              & (1, 13, 8235), (2, 2, 0), (2, 3, 0), (3, 1, 0)       \\
%     5490                                           & (1, 12, 13725)                              & (2, 2, 2745), (2, 3, 0), (3, 1, 0)                   \\
%     8235                                           & (1, 12, 16470)                              & (1, 13, 13725), (2, 2, 2745), (2, 3, 0), (3, 1, 0)   \\
%     10980                                          & (1, 12, 16470)                              & (1, 13, 13725), (2, 2, 2745), (2, 3, 0), (3, 1, 0)   \\
%     13725                                          & (1, 12, 16470)                              & (1, 13, 13725), (2, 2, 2745), (2, 3, 0), (3, 1, 0)   \\
%     16470                                          & (1, 13, 16470)                              & (2, 2, 2745), (2, 3, 0), (3, 1, 0)                   \\
%     19215                                          & (1, 12, 16470)                              & (1, 13, 13725), (2, 2, 2745), (2, 3, 0), (3, 1, 0)   \\
%     21960                                          & (1, 12, 16470)                              & (1, 13, 13725), (2, 2, 2745), (2, 3, 0), (3, 1, 0)   \\
%     24705                                          & (1, 12, 16470)                              & (1, 13, 13725), (2, 2, 2745), (2, 3, 0), (3, 1, 0)   \\
%     27450                                          & (1, 12, 16470)                              & (1, 13, 13725), (2, 2, 2745), (2, 3, 0), (3, 1, 0)   \\
%     30195                                          & (2, 2, 2745)                                & (2, 3, 0), (3, 1, 0)                                 \\
%     32940                                          & (1, 13, 16470)                              & (2, 2, 2745), (2, 3, 0), (3, 1, 0)                   \\
%     35685                                          & (1, 13, 13725)                              & (2, 2, 2745), (2, 3, 0), (3, 1, 0)                   \\
%     38430                                          & (1, 13, 10980)                              & (2, 2, 2745), (2, 3, 0), (3, 1, 0)                   \\
%     41175                                          & (1, 12, 13725)                              & (1, 13, 10980), (2, 2, 2745), (2, 3, 0), (3, 1, 0)   \\
%     43920                                          & (1, 13, 10980)                              & (2, 2, 2745), (2, 3, 0), (3, 1, 0)                   \\
%     46665                                          & (2, 2, 2745)                                & (2, 3, 0), (3, 1, 0)                                 \\
%     49410                                          & (2, 2, 2745)                                & (2, 3, 0), (3, 1, 0)                                 \\
%     52155                                          & (1, 12, 16470)                              & (1, 13, 13725), (2, 2, 2745), (2, 3, 0), (3, 1, 0)   \\
%     54900                                          & (1, 13, 13725)                              & (2, 2, 2745), (2, 3, 0), (3, 1, 0)                   \\
%     57645                                          & (1, 13, 13725)                              & (2, 2, 2745), (2, 3, 0), (3, 1, 0)                   \\
%     60390                                          & (1, 12, 13725)                              & (2, 2, 2745), (2, 3, 0), (3, 1, 0)                   \\
%     63135                                          & (1, 13, 16470)                              & (2, 2, 2745), (2, 3, 0), (3, 1, 0)                   \\
%     65880                                          & (1, 13, 16470)                              & (2, 2, 2745), (2, 3, 0), (3, 1, 0)                   \\
%     68625                                          & (2, 2, 2745)                                & (2, 3, 0), (3, 1, 0)                                 \\
%     71370                                          & (1, 13, 13725)                              & (2, 2, 2745), (2, 3, 0), (3, 1, 0)                   \\
%     74115                                          & (1, 12, 13725)                              & (2, 2, 2745), (2, 3, 0), (3, 1, 0)                   \\
%     76860                                          & (1, 13, 13725)                              & (2, 2, 2745), (2, 3, 0), (3, 1, 0)                   \\
%     79605                                          & (1, 13, 13725)                              & (2, 2, 2745), (2, 3, 0), (3, 1, 0)                   \\
%     82350                                          & (1, 12, 13725)                              & (2, 2, 2745), (2, 3, 0), (3, 1, 0)                   \\
%     85095                                          & (1, 12, 13725)                              & (1, 13, 10980), (2, 2, 2745), (2, 3, 0), (3, 1, 0)   \\
%     87840                                          & (1, 13, 16470)                              & (2, 2, 2745), (2, 3, 0), (3, 1, 0)                   \\
%     90585                                          & (1, 13, 16470)                              & (2, 2, 2745), (2, 3, 0), (3, 1, 0)                   \\
%     93330                                          & (1, 13, 13725)                              & (2, 2, 2745), (2, 3, 0), (3, 1, 0)                   \\
%     96075                                          & (1, 13, 16470)                              & (2, 2, 2745), (2, 3, 0), (3, 1, 0)                   \\
%     98820                                          & (1, 13, 16470)                              & (2, 2, 2745), (2, 3, 0), (3, 1, 0)                   \\
%     101565                                         & (1, 13, 13725)                              & (2, 2, 2745), (2, 3, 0), (3, 1, 0)                   \\
%     104310                                         & (1, 13, 16470)                              & (2, 2, 2745), (2, 3, 0), (3, 1, 0)                   \\
%     107055                                         & (1, 13, 13725)                              & (2, 2, 2745), (2, 3, 0), (3, 1, 0)                   \\
%     109800                                         & (1, 13, 13725)                              & (2, 2, 2745), (2, 3, 0), (3, 1, 0)                   \\
%     112545                                         & (1, 12, 16470)                              & (1, 13, 13725), (2, 2, 2745), (2, 3, 0), (3, 1, 0)   \\
%     115290                                         & (1, 13, 16470)                              & (2, 2, 2745), (2, 3, 0), (3, 1, 0)                   \\
%     118035                                         & (1, 13, 13725)                              & (2, 2, 2745), (2, 3, 0), (3, 1, 0)                   \\
%     120780                                         & (1, 13, 16470)                              & (2, 2, 2745), (2, 3, 0), (3, 1, 0)                   \\
%     123525                                         & (1, 13, 13725)                              & (2, 2, 2745), (2, 3, 0), (3, 1, 0)                   \\
%     126270                                         & (1, 12, 16470)                              & (1, 13, 13725), (2, 2, 2745), (2, 3, 0), (3, 1, 0)   \\
%     129015                                         & (2, 2, 2745)                                & (2, 3, 0), (3, 1, 0)                                 \\
%     131760                                         & (2, 2, 2745)                                & (2, 3, 0), (3, 1, 0)                                 \\
%     134505                                         & (1, 13, 16470)                              & (2, 2, 2745), (2, 3, 0), (3, 1, 0)                   \\
%     137250                                         & (1, 13, 13725)                              & (2, 2, 2745), (2, 3, 0), (3, 1, 0)                   \\
%     139995                                         & (2, 2, 2745)                                & (2, 3, 0), (3, 1, 0)                                 \\
%     142740                                         & (2, 2, 2745)                                & (2, 3, 0), (3, 1, 0)                                 \\
%     145485                                         & (1, 12, 16470)                              & (1, 13, 13725), (2, 2, 2745), (2, 3, 0), (3, 1, 0)   \\
%     148230                                         & (2, 2, 2745)                                & (2, 3, 0), (3, 1, 0)                                 \\
%     150975                                         & (1, 13, 16470)                              & (2, 2, 2745), (2, 3, 0), (3, 1, 0)                   \\
%     153720                                         & (1, 12, 13725)                              & (2, 2, 2745), (2, 3, 0), (3, 1, 0)                   \\
%     156465                                         & (1, 13, 13725)                              & (2, 2, 2745), (2, 3, 0), (3, 1, 0)                   \\
%     159210                                         & (1, 13, 13725)                              & (2, 2, 2745), (2, 3, 0), (3, 1, 0)                   \\
%     161955                                         & (1, 13, 16470)                              & (2, 2, 2745), (2, 3, 0), (3, 1, 0)                   \\
%     164700                                         & (1, 13, 13725)                              & (2, 2, 2745), (2, 3, 0), (3, 1, 0)                   \\
% \end{longtable}
% \end{center



% \begin{landscape}
%     \begin{table}[h]
%         \centering
%         \begin{tabular}{cccccc}
%             \toprule
%             \multicolumn{1}{c}{} & \multicolumn{3}{c}{\textbf{Topic 2}} & \multicolumn{2}{c}{\textbf{Topic 3}}                   \\
%             \cmidrule(rl){2-4} \cmidrule(rl){5-6}
%             \textbf{Topic 1}     & {A}                                  & {B}                                  & {C} & {D} & {E} \\
%             \midrule
%             \rowcolor{lavender}
%             row1                 & A.1                                  & B.1                                  & C.1 & D.1 & E.1 \\
%             row2                 & A.2                                  & B.2                                  & C.2 & D.2 & E.2 \\
%             \rowcolor{lavender}
%             row3                 & A.3                                  & B.3                                  & C.3 & D.3 & E.3 \\
%             row4                 & A.4                                  & B.4                                  & C.4 & D.4 & E.4 \\
%             \rowcolor{lavender}
%             row5                 & A.5                                  & B.5                                  & C.5 & D.5 & E.5 \\
%             \bottomrule
%         \end{tabular}
%     \end{table}
% \end{landscape}


% \begin{minted}{python}
%     \caption{Example of a listing.}
%     \label{lst:example}

%     def boring(args = None):
%     pass
% \end{minted}


